% section — subsection — subsubsection — paragraph — subparagraph

\section{算法步骤}\label{section:最终结论}
下面总结3轴融合,6轴融合,9轴融合的步骤。
\subsection{3轴融合}
3轴融合算法可以分解为一下步骤:
\begin{enumerate}
    \item 四元数初始化

        积分操作需要初始值,而四元数的几何意义不是很明确,难以确定其初始值.故四元数的初始值需要通过欧拉角计算.初始欧拉角全为零,代入式\ref{欧拉角转四元数},可以得到四元数的初值.
    \item 微积分求四元数

        通过陀螺仪输出的角速度求四元数微分,利用该微分值求四元数.四元数微积分公式为式\ref{四元数递推分量方程}.
    \item 计算瞬时欧拉角

        为了和加计,磁计融合以及便于后面的姿态控制,四元数需要转换为有物理含义的欧拉角.四元数转欧拉角使用公式\ref{四元数转欧拉角}.
\end{enumerate}

\subsection{6轴融合}
6轴融合

\subsection{9轴融合}
9轴融合


