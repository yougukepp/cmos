

\subsubsection{四元数积分方程}
\begin{equation}\label{四元数积分方程}
    ^s_n\hat{q}_{est,t}=^s_n\hat{q}_{est,t-1}+^s_n\dot{q}_{\omega,t}\Delta t
\end{equation} 

$^s_n\hat{q}_{est,t}$    代表本次迭代计算出的姿态,该量为姿态解算的结果。
$^s_n\hat{q}_{est,t-1}$  代表上次迭代计算出的姿态,该量已知。
$^s_n\dot{q}_{\omega,t}$ 代表使用陀螺仪数据计算出的姿态微分,该量未知。
$\Delta t$               代表上次与本次迭代的时间间隔,该量为算法的可调参数,通常越小越好。
下面推导$^s_n\dot{q}_{\omega,t}$的计算公式,利用四元数微分方程:
\begin{equation}\label{四元数微分方程}
    ^s_n\dot{q}_{\omega,t}=\frac{1}{2} ^s_n\hat{q}_{est,t-1} \cdot ^s\omega_t
\end{equation} 

$^s_n\dot{q}_{\omega,t}$ 代表使用陀螺仪数据计算出的姿态微分,与式\ref{四元数积分方程}中含义相同,需要求解。
$^s_n\hat{q}_{est,t-1}$  代表上次姿态值\footnote{3轴融合算法中仅用陀螺仪数据估计,6轴或9轴使用额外的传感器数据。}
$\cdot$                  代表四元数乘法。
$^s\omega_t$             代表陀螺仪测量出的角速度\footnote{基于s系}。

四元数微分方程的推导笔者不会,如果想自己推导究可以研究参考文献\citet{四元数微分方程的推导}。由于C语言中没有向量的概念,所以式\ref{四元数积分方程}和\ref{四元数微分方程}需要转换为分量形式。

\newpage
华丽的分割页
\newpage













$^s_n\dot{q}_{\omega,t-1}$表示上次迭代的姿态微分,坐标表示从s系转为n系,$\omega$表示陀螺仪器获取的姿态\footnote{$\omega$下标用于姿态微分,$est$下标用于姿态}。$\Delta t$表示时间间隔。通常$\Delta t$非常小,
\begin{equation}\label{近似1}
^s_n\dot{q}_{\omega,t}=^s_n\dot{q}_{\omega,t-1}
\end{equation}
所以为了方便计算
\begin{equation}\label{四元数积分方程2}
    ^s_n\hat{q}_{est,t}={^s_n\hat{q}_{est,t-1}}+^s_n\dot{q}_{\omega,t}\Delta t
\end{equation}

$\Delta t$

$^s_n\dot{q}_{est,t-1}$表示t时刻从s系旋转到n系的四元数

\footnote{可以理解为四元数是一个时间的函数$^s_n{q}=func(t)$,$^s_n\dot{q}=func'(t)$}微分\footnote{下标$\omega$表示该值通过陀螺仪输出的角速度算出},${^s_n\hat{q}_{est,t-1}}$表示t-1\footnote{上次迭代的时刻}时刻姿态的估计值,也就是s系到n系四元数的估计值\footnote{est是estimation缩写}。$\otimes$表示四元数乘法

式\ref{四元数积分方程}中$\Delta t$表示迭代周期。

用四元数表示表示向量,将其标量部分插入零即可。所以陀螺仪提供的数据为:
\begin{eqnarray}
    ^s\omega=[0\quad\omega_{x}\quad\omega_{y}\quad\omega_{z}] \label{陀螺仪数据}
\end{eqnarray}

公式中$^s\omega$表示在传感器坐标系中陀螺仪测到的角速度.
$\omega_{x},\omega_{y},\omega_{z}$分别表示飞行器绕s系的前、右、上轴的角速度。

四元数微分方程为:
\begin{eqnarray}
    ^s_n\dot{q}_{\omega,t}=\frac{1}{2}{^s_n\hat{q}_{est,t-1}}\otimes^s\omega_t \label{四元数微分方程}
\end{eqnarray}

四元数微分方程使用就行,这里不推导了,笔者也不会。
$^s_n\dot{q}_{\omega,t}$表示t时刻从s系旋转到n系的四元数\footnote{可以理解为四元数是一个时间的函数$^s_n{q}=func(t)$,$^s_n\dot{q}=func'(t)$}微分\footnote{下标$\omega$表示该值通过陀螺仪输出的角速度算出},${^s_n\hat{q}_{est,t-1}}$表示t-1\footnote{上次迭代的时刻}时刻姿态的估计值,也就是s系到n系四元数的估计值\footnote{est是estimation缩写}。$\otimes$表示四元数乘法

四元数写为分量形式:
\begin{eqnarray}
    \hat{q}=[\hat{q}_0\quad \hat{q}_1\quad \hat{q}_2\quad \hat{q}_3] \label{分量四元数}
\end{eqnarray}

在进行四元数微分方程分量化前定义虚数运算公式:
\begin{eqnarray}
    i^2 = j^2 = k^2 = -1
\end{eqnarray}
\begin{eqnarray}
    ij = -ji = k
\end{eqnarray}
\begin{eqnarray}
    jk = -kj = i
\end{eqnarray}
\begin{eqnarray}
    ki = -ik = j
\end{eqnarray}

四元数微分方程分量化推导:
\begin{equation} \label{微分方程分量推导}
    \begin{aligned}
        {^s_n\hat{q}_{est,t-1}}\otimes^s\omega_t
        = &[\hat{q}_0\quad i\hat{q}_1\quad j\hat{q}_2\quad k\hat{q}_3] \otimes [0\quad i\omega_{x}\quad j\omega_{y}\quad k\omega_{z}] \\
        = & \hat{q}_0 \cdot 0 + i    \hat{q}_0 \omega_x + j    \hat{q}_0 \omega_y + k    \hat{q}_0 \omega_z \\
      + i & \hat{q}_1 \cdot 0 + (ii) \hat{q}_1 \omega_x + (ij) \hat{q}_1 \omega_y + (ik) \hat{q}_1 \omega_z \\
      + j & \hat{q}_2 \cdot 0 + (ji) \hat{q}_2 \omega_x + (jj) \hat{q}_2 \omega_y + (jk) \hat{q}_2 \omega_z \\
      + k & \hat{q}_3 \cdot 0 + (ki) \hat{q}_3 \omega_x + (kj) \hat{q}_3 \omega_y + (kk) \hat{q}_3 \omega_z \\
    =   i & \hat{q}_0 \omega_x + j \hat{q}_0 \omega_y + k \hat{q}_0 \omega_z \\
        - & \hat{q}_1 \omega_x + k \hat{q}_1 \omega_y - j \hat{q}_1 \omega_z \\
      - k & \hat{q}_2 \omega_x -   \hat{q}_2 \omega_y + i \hat{q}_2 \omega_z \\
        + j & \hat{q}_3 \omega_x - i \hat{q}_3 \omega_y -   \hat{q}_3 \omega_z \\
        = - & (\hat{q}_1 \omega_x + \hat{q}_2 \omega_y + \hat{q}_3 \omega_z) \\
        + i &(\hat{q}_0 \omega_x + \hat{q}_2 \omega_z - \hat{q}_3 \omega_y) \\
        + j & (\hat{q}_0 \omega_y - \hat{q}_1 \omega_z + \hat{q}_3 \omega_x) \\
        + k & (\hat{q}_0 \omega_z + \hat{q}_1 \omega_y - \hat{q}_2 \omega_x)
    \end{aligned}
\end{equation}

整理式\ref{四元数微分方程}和式\ref{微分方程分量推导},可得微分方程分量形式:
\begin{eqnarray}
    \begin{aligned}
    ^s_n\dot{q}_{\omega,t}&=\frac{1}{2}{^s_n\hat{q}_{est,t-1}}\otimes^s\omega_t
    =\frac{1}{2}
    \left[\begin{array}{c}
            - \hat{q}_1\omega_x - \hat{q}_2 \omega_y - \hat{q}_3 \omega_z \\
              \hat{q}_0\omega_x + \hat{q}_2 \omega_z - \hat{q}_3 \omega_y \\
              \hat{q}_0\omega_y - \hat{q}_1 \omega_z + \hat{q}_3 \omega_x \\
              \hat{q}_0\omega_z + \hat{q}_1 \omega_y - \hat{q}_2 \omega_x
    \end{array}\right] \\
    &=\frac{1}{2}
    \left[ \begin{array}{rrr}
            -\hat{q}_{1,t-1} & -\hat{q}_{2,t-1} & -\hat{q}_{3,t-1} \\
             \hat{q}_{0,t-1} & -\hat{q}_{3,t-1} &  \hat{q}_{2,t-1} \\
             \hat{q}_{3,t-1} &  \hat{q}_{0,t-1} & -\hat{q}_{1,t-1} \\
            -\hat{q}_{2,t-1} &  \hat{q}_{1,t-1} &  \hat{q}_{0,t-1}
    \end{array} \right]
    \left[\begin{array}{c}
            \omega_{x} \\
            \omega_{y} \\
            \omega_{z}
    \end{array}\right] \label{微分最终式}
    \end{aligned}
\end{eqnarray}

式\ref{微分最终式}中四元数上方的三角表示估计值,上方点表示微分,下标t表示当前迭代时刻,下标t-1表示上次迭代时刻,下标0、1、2、3表示四元数分量编号。

整理上文的推导,可以写出三轴姿态算法的分量表示步骤:
\begin{enumerate}
    \item 姿态初始化

        算法启动时使s系前右上与n系北东天分别重合。所以欧拉角:
        \begin{equation}\label{初始欧拉角}
            \psi = \theta = \phi = 0
        \end{equation}
        %
        % q0 = cos(c/2)cos(b/2)cos(a/2) + sin(c/2)sin(b/2)sin(a/2)
        % q1 = sin(c/2)cos(b/2)cos(a/2) - cos(c/2)sin(b/2)sin(a/2)
        % q2 = cos(c/2)sin(b/2)cos(a/2) + sin(c/2)cos(b/2)sin(a/2) 
        % q3 = cos(c/2)cos(b/2)sin(a/2) + sin(c/2)sin(b/2)cos(a/2)
        % 初始化时 a b c 都为0带入上公式 所以初始化为 s_quaternion[4] = {1,0,0,0}
        %利用式\ref{初始欧拉角}和式\ref{欧拉角转四元数},得出四元数初始值为\footnote{研究推导}:
        利用式\ref{初始欧拉角}和式,得出四元数初始值为\footnote{推导欧拉角转四元数公式}:
        \begin{equation}\label{四元数初值}
            \begin{aligned} 
                q_{0,0} = 1 \\
                q_{1,0} = 0 \\
                q_{2,0} = 0 \\
                q_{3,0} = 0
            \end{aligned}
        \end{equation}
        式\ref{四元数初值}中第二个下标表示初始值。
    \item 微分

        整理式\ref{微分最终式},得出微分迭代方程:
        \begin{equation} \label{微分迭代方程}
            \begin{aligned} 
                &\dot{q}_{0,t} = -\hat{q}_{1,t-1} \omega_x - \hat{q}_{2,t-1} \omega_y - \hat{q}_{3,t-1} \omega_z \\
                &\dot{q}_{1,t} =  \hat{q}_{0,t-1} \omega_x - \hat{q}_{3,t-1} \omega_y + \hat{q}_{2,t-1} \omega_z \\
                &\dot{q}_{2,t} =  \hat{q}_{3,t-1} \omega_x + \hat{q}_{0,t-1} \omega_y - \hat{q}_{1,t-1} \omega_z \\
                &\dot{q}_{3,t} = -\hat{q}_{2,t-1} \omega_x + \hat{q}_{1,t-1} \omega_y + \hat{q}_{0,t-1} \omega_z
            \end{aligned}
        \end{equation} 
        式\ref{微分迭代方程}中四元数上方的点表示四元数微分\footnote{当前迭代想对于上次迭代时的变化量},下标t表示当前迭代时刻,下标t-1表示上次迭代时刻,下标0、1、2、3表示四元数分量编号,$x,y,z$代表s系前右上
    \item 积分

        整理式\ref{四元数积分方程}和\ref{微分迭代方程},积分即可求得当前姿态。
        \begin{equation} \label{积分迭代方程}
            \begin{aligned} 
                &\dot{q}_{0,t} = \hat{q}_{0,t-1} + \dot{q}_{0,t} \Delta t \\
                &\dot{q}_{1,t} = \hat{q}_{1,t-1} + \dot{q}_{1,t} \Delta t \\
                &\dot{q}_{2,t} = \hat{q}_{2,t-1} + \dot{q}_{2,t} \Delta t \\
                &\dot{q}_{3,t} = \hat{q}_{3,t-1} + \dot{q}_{3,t} \Delta t
            \end{aligned}
        \end{equation} 
        式\ref{积分迭代方程}中上下标含义与式\ref{微分迭代方程}相同
\end{enumerate}

以上所有公式都转换为了分量形式,可以直接使用C语言实现陀螺仪的3轴姿态算法。

\section{四元数转欧拉角}
四元数转欧拉角推导

欧拉角转四元数推导

\section{后续工作}
\begin{enumerate}
    \item 考虑“指东针“

        将重力场矢量与地磁叉乘得到一个仅有东向\footnote{或向西,由叉乘顺序决定}分量的矢量,这样可以简化计算。但有两个问题需要验证:
        \begin{enumerate}
            \item 叉乘的代价与简化的计算量相比是否值得;
            \item 叉乘后该量比不是磁力计独立测试的数据,与加速度计有一定的相关性,是否会引入误差。
        \end{enumerate} 
    \item 卡尔曼滤波

        温度计和房间温度的例子
    \item 加速度标定

        取平均或最小二乘法
    \item 控制算法

        PID算法
\end{enumerate}

