% section — subsection — subsubsection — paragraph — subparagraph

\section{欧拉角与方向余弦}
首先分析二维位旋转.
\begin{figure}[!hbp]
    \begin{center}
        \includegraphics[height=5cm, width=5cm]{fig/2dxoyrotate.pdf}
        \caption{二维旋转}\label{二维旋转}
    \end{center}
\end{figure}

根据图\ref{二维旋转}容易得出:
\begin{equation*}
    \bfx{p}=x\bfi+y\bfj,
    \bfx{p}=x'\bfxd{i}+y'\bfxd{j} \\ 
\end{equation*} 

进行如下变换
\begin{equation*}
    \begin{split}
                    & \bfx{p}=x\bfi+y\bfj, \bfx{p}=x'\bfxd{i}+y'\bfxd{j} \\
        \Rightarrow & x\bfi+y\bfj=x'\bfxd{i}+y'\bfxd{j} \\
        \Rightarrow & x\bfi\cdot\bfi+y\bfi\cdot\bfj=x'\bfi\cdot\bfxd{i}+y'\bfi\cdot\bfxd{j} \\
        \Rightarrow & x=\bfi\cdot\bfxd{i}x'+\bfi\cdot\bfxd{j}y'
    \end{split}
\end{equation*} 

同理:
\begin{equation*}
    \begin{split}
                    & x\bfi+y\bfj=x'\bfxd{i}+y'\bfxd{j} \\
        \Rightarrow & x\bfj\cdot\bfi+y\bfj\cdot\bfj=x'\bfj\cdot\bfxd{i}+y'\bfj\cdot\bfxd{j} \\
        \Rightarrow & y=\bfj\cdot\bfxd{i}x'+\bfj\cdot\bfxd{j}y'
    \end{split}
\end{equation*} 

综合以上推导可以得出:
\begin{equation}\label{二维方向余弦阵}
    \begin{split}
        \left[\begin{matrix}
                x' \\
                y'
        \end{matrix}\right]
        =
        \left[\begin{matrix}
                \bfi\bfid & \bfj\bfid \\
                \bfi\bfjd & \bfj\bfjd
        \end{matrix}\right]
        \left[\begin{matrix}
                x \\
                y
        \end{matrix}\right]
    \end{split}
\end{equation}

其中$x$,$y$     表示$\bfx{p}$在原坐标系坐标;
$\bfi$,$\bfj$   表示在原坐标系坐标轴方向单位向量;
$\bfid$,$\bfjd$ 表示在旋转后坐标系坐标轴方向单位向量;
$x'$,$y'$       表示$\bfx{p}$在旋转后坐标系坐标.类似二维旋转,三维旋转方向余弦阵可以表示为:
\begin{equation}\label{三维方向余弦阵}
    \begin{split}
        \left[\begin{matrix}
                x' \\
                y' \\
                z'
        \end{matrix}\right]
        =
        \left[\begin{matrix}
                \bfi\bfid & \bfj\bfid & \bfk\bfid \\
                \bfi\bfjd & \bfj\bfjd & \bfk\bfjd \\
                \bfi\bfkd & \bfj\bfkd & \bfk\bfkd
        \end{matrix}\right]
        \left[\begin{matrix}
                x \\
                y \\
                z
        \end{matrix}\right]
    \end{split}
\end{equation}

其中$x$,$y$,$z$         表示$\bfx{p}$在原坐标系坐标;
$\bfi$,$\bfj$,$\bfk$    表示在原坐标系坐标轴方向单位向量;
$\bfid$,$\bfjd$,$\bfkd$ 表示在旋转后坐标系坐标轴方向单位向量;
$x'$,$y'$,$z'$          表示$\bfx{p}$在旋转后坐标系坐标. 这里有一点需要强调,本文坐标系统一采用\textcolor[rgb]{1,0,0}{右手坐标系},结合绘图的美观性,所有旋转轴\footnote{旋转所绕的轴,而非旋转的轴}指向读者.

\subsection{三次旋转}
本文绕坐标轴旋转的顺序为$z$-$y$-$x$,首先绕$z$轴旋转$\psi$:
$\phi$
$\theta$

