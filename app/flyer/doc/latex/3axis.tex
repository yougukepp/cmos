
下面对三个算法分别阐述。
\subsection{3轴姿态}
3轴姿态使用陀螺仪的数据\footnote{角速度}计算姿态的微分,然后对微分求积分获取当前姿态,下面开始分析。
\subsubsection{四元数微积分}
\begin{equation}\label{四元数积分方程}
    ^s_n\bm{q}_{est,t}=^s_n\bm{q}_{est,t-1}+^s_n\dot{{\bm{q}}}_{\omega,t} \Delta t
\end{equation} 

$^s_n\bm{q}_{est,t}$            代表本次迭代计算出的姿态,该量为姿态解算的结果;
$^s_n\bm{q}_{est,t-1}$          代表上次迭代计算出的姿态,该量已知;
$^s_n\dot{{\bm{q}}}_{\omega,t}$ 代表使用陀螺仪数据计算出的姿态微分,该量未知;
$\Delta t$                      代表上次与本次迭代的时间间隔,该量为算法的可调参数,通常越小越好。

$^s_n\dot{{\bm{q}}}_{\omega,t}$的计算公式由四元数微分方程计算:
\begin{equation}\label{四元数微分方程}
    ^s_n\dot{{\bm{q}}}_{\omega,t}=\frac{1}{2}{^s_n\bm{q}_{est,t-1}}\cdot^s{\bm{\omega}}_t
\end{equation} 

$^s_n\dot{{\bm{q}}}_{\omega,t}$ 代表使用陀螺仪数据计算出的姿态微分,与式\ref{四元数积分方程}中含义相同,需要求解。
$^s_n\bm{q}_{est,t-1}$          代表上次姿态值\footnote{3轴融合算法中仅用陀螺仪数据估计,6轴或9轴使用额外的传感器数据。}
$\cdot$                         代表四元数乘法。
$^s\bm{\omega}_t$               代表陀螺仪测量出的角速度\footnote{基于s系}。

四元数微分方程的推导笔者不会,如果想自己推导究可以研究参考文献\citet{四元数微分方程的推导}。由于C语言中没有向量的概念,所以式\ref{四元数积分方程}和\ref{四元数微分方程}需要转换为分量形式。
\subsubsection{四元数分量微积分}
式\ref{四元数积分方程}的分量形式为:
\begin{eqnarray}\label{四元数分量积分方程}
    \begin{split}
        q_{0,t} = q_{0,t-1} + \dot{q}_{0,t} \cdot \Delta t \\ 
        q_{1,t} = q_{1,t-1} + \dot{q}_{1,t} \cdot \Delta t \\
        q_{2,t} = q_{2,t-1} + \dot{q}_{2,t} \cdot \Delta t \\
        q_{3,t} = q_{3,t-1} + \dot{q}_{3,t} \cdot \Delta t
    \end{split}
\end{eqnarray} 

$q_{x,t}$       代表当前第$x$个姿态分量的值;
$q_{x,t-1}$     代表上次迭代得到第$x$个姿态分量值;
$\dot{q}_{x,t}$ 代表当前第$x$个姿态分量的微分;
$\Delta t$      代表上次与本次迭代的时间间隔,该量为算法的可调参数。

式\ref{四元数分量积分方程}中的每个分量利用上次迭代的分量值和微分求解当前值,公式右边的$\dot{q}_{x,t}$需要使用式\ref{四元数分量微分方程}求解:
\begin{eqnarray}\label{四元数分量微分方程}
    \begin{split}
        \dot{q}_{0,t}&=- &\frac{1}{2}(q_{1,t-1}\omega_x+q_{2,t-1}\omega_y+q_{3,t-1}\omega_z) \\
        \dot{q}_{1,t}&=  &\frac{1}{2}(q_{0,t-1}\omega_x+q_{2,t-1}\omega_z-q_{3,t-1}\omega_y) \\
        \dot{q}_{2,t}&=  &\frac{1}{2}(q_{0,t-1}\omega_y-q_{1,t-1}\omega_z+q_{3,t-1}\omega_x) \\
        \dot{q}_{3,t}&=  &\frac{1}{2}(q_{0,t-1}\omega_z+q_{1,t-1}\omega_y-q_{2,t-1}\omega_x)
    \end{split}
\end{eqnarray} 

式\ref{四元数分量微分方程}是式\ref{四元数微分方程}的分量形式,\ref{section:四元数分量微分方程推导}完成式四元数分量微分方程的推导。
$\dot{q}_{x,t}$ 代表当前第$x$个姿态分量的微分值; 
$q_{x,t-1}$     代表上次迭代得到第$x$个姿态分量值;
$\omega_x$      代表陀螺仪测量的$x$\footnote{前}轴的角速度;
$\omega_y$      代表陀螺仪测量的$y$\footnote{右}轴的角速度;
$\omega_z$      代表陀螺仪测量的$z$\footnote{天}轴的角速度。

\subsubsection{四元数分量微分方程推导}\label{section:四元数分量微分方程推导}
三维矢量表示为四元数时,四元数标量分量为0,矢量的三维分量与四元数虚部对应。所以陀螺仪测量出$^s\bm{\omega}_t$的角速度矢量与姿态矢量分别表示为四元数:
\begin{eqnarray}\label{四元数分量方程}
    \begin{split}
        ^s_n\bm{q}_t&=[q_{0,t}\quad q_{1,t}\quad q_{2,t}\quad q_{3,t}] \\
        ^s\bm{\omega}_t&=[0\quad \omega_x\quad \omega_y\quad \omega_z]
    \end{split}
\end{eqnarray} 

式\ref{四元数分量方程}代入\ref{四元数乘法展开}写为分量形式得:
\begin{eqnarray}\label{四元数积}
    \begin{split} 
        ^s_n\bm{q}_{est,t-1}\cdot^s{\bm{\omega}}_t=
        \left[\begin{matrix}
                -q_{1,t-1}\omega_x-q_{2,t-1}\omega_y-q_{3,t-1}\omega_z \\
                 q_{0,t-1}\omega_x+q_{2,t-1}\omega_z-q_{3,t-1}\omega_y \\
                 q_{0,t-1}\omega_y-q_{1,t-1}\omega_z+q_{3,t-1}\omega_x \\
                 q_{0,t-1}\omega_z+q_{1,t-1}\omega_y-q_{2,t-1}\omega_x
        \end{matrix}\right]^\mathrm{T}
        \left[\begin{matrix}
                1 \\ i \\ j \\ k
        \end{matrix}\right] \\
    \end{split}
\end{eqnarray} 

式\ref{四元数积}代入式\ref{四元数微分方程},并写为分量形式得到式\ref{四元数分量微分方程}。

\subsubsection{3轴姿态解算小结}\label{section:3轴姿态解算小结}
\begin{enumerate}
        \item 初始化 
            
            首先,将初始欧拉角$\psi=\theta=\phi=0$代入公式\ref{欧拉角转四元数}得到初始四元数为:
            \begin{equation}\label{四元数初始化}
                ^s_n\bm{q}=[1\quad 0\quad 0\quad 0]^\mathrm{T}
            \end{equation} 

        \item 求姿态微分\label{3轴迭代开始}
            
            使用公式\ref{四元数分量微分方程}求解当前姿态微分;

        \item 积分求姿态
            
            使用公式\ref{四元数分量积分方程}求解当前姿态; 

        \item 归一化 
            
            使用公式\ref{四元数归一化}归一化姿态并进入下次一迭代\ref{3轴迭代开始}。
\end{enumerate}

