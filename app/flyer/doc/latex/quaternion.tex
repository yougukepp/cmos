% section — subsection — subsubsection — paragraph — subparagraph

\section{四元数推导}
四元数表示姿态时仅需要4个变量,比方向余弦矩阵表示姿态\footnote{需要9个变量}少需要5个变量,同时也没有欧拉角的万向锁问题\footnote{我也不懂}.下面分析四元数的性质为算法设计打基础.

\subsection{定义}
四元数的定义如下:
\begin{equation}\label{四元数定义}
    Q = q_0 + q_1\bfi + q_2\bfj + q_3\bfk
\end{equation} 

其中$Q$     表示四元数,
$q_x$       表示四元数的第$x$个分量,
$\bfi$,$\bfj$,$bfk$表示虚部.他们的关系为:
\begin{equation}\label{四元数虚部关系}
    \begin{split}
        & \bfi \cdot \bfi =  \bfj \cdot \bfj = \bfk \cdot \bfk = -1 \\
        & \bfi \cdot \bfj = -\bfj \cdot \bfi = \bfk \\
        & \bfj \cdot \bfk = -\bfk \cdot \bfj = \bfi \\
        & \bfk \cdot \bfi = -\bfi \cdot \bfk = \bfj
    \end{split}
\end{equation} 

单位四元数{归一化后}的定义如下:
\begin{equation}\label{四元数归一化定义}
    \begin{split}
        Q = & \costhetadtwo + \bfx{n} \cdot \sinthetadtwo \\
        \bfx{n} = & \cos{\alpha}i + \cos{\beta}j + \cos{\gamma}k
    \end{split}
\end{equation} 

其中$\theta$    表示旋转的角度,
$\bfx{n}$       表示旋转轴向量,
$\alpha$        表示旋转轴与参考系$x$轴夹角,
$\beta$         表示旋转轴与参考系$y$轴夹角,
$\gamma$        表示旋转轴与参考系$z$轴夹角.

\subsection{欧拉角转四元数}
本文采用的欧拉角顺序为$Z$-$Y$-$X$,所以可以用三次旋转表示\cite{飞行器专题资料}旋转\footnote{s系下推导}.
\begin{equation}\label{四元数三次旋转}
    Q_{xyz} = Q_z \cdot Q_y \cdot Q_x = (\cospsidtwo + \sinpsidtwo \bfk) \cdot (\cosphidtwo + \sinphidtwo \bfj) \cdot (\costhetadtwo + \sinthetadtwo \bfi)
\end{equation} 

其中$Q_{xyz}$   表示旋转四元数,
$Q_{z}$         表示绕$z$轴的旋转四元数,
$Q_{y}$         表示绕$y$轴的旋转四元数,
$Q_{x}$         表示绕$x$轴的旋转四元数,
$\theta$        表示横滚角,
$\phi$          表示俯仰角,
$\psi$          表示偏航角.将式\ref{四元数三次旋转}展开\footnote{每个括号提一项拼尾巴$1$,$\bfi$,$\bfj$,$\bfk$}可以得到:
\begin{equation}\label{欧拉角转四元数}
    \begin{split}
        \left[\begin{matrix}
                q_0 \\
                q_1 \\
                q_2 \\
                q_3
        \end{matrix}\right]\transpose
        =
        \left[\begin{matrix}
                \costhetadtwo\cosphidtwo\cospsidtwo + \sinthetadtwo\sinphidtwo\sinpsidtwo \\
                \sinthetadtwo\cosphidtwo\cospsidtwo - \costhetadtwo\sinphidtwo\sinpsidtwo \\
                \costhetadtwo\sinphidtwo\cospsidtwo + \sinthetadtwo\cosphidtwo\sinpsidtwo \\
                \costhetadtwo\cosphidtwo\sinpsidtwo - \sinthetadtwo\sinphidtwo\cospsidtwo
        \end{matrix}\right]\transpose
        \left[\begin{matrix}
                1 \\
                \bfi \\
                \bfj \\
                \bfk \\
        \end{matrix}\right]
    \end{split}
\end{equation} 

其中$\theta$    表示横滚角,
$\phi$          表示俯仰角,
$\psi$          表示偏航角,
$q_x$           表示四元数的第$x$个分量.式\ref{欧拉角转四元数}是欧拉角转换为四元数的公式.

\subsection{四元数转欧拉角}
由于笔者尚未发现四元数直接转换欧拉角的推导,所以这里先推导四元数转方向余弦阵,再利用方向余弦阵推到四元数转欧拉角的公式.
\subsubsection{四元数转方向余弦阵}
设矢量$\bfx{R}=x\bfi+y\bfj+z\bfk$,在\textcolor[rgb]{1,0,0}{旋转后的坐标系}\footnote{本文的旋转在未说明的情况下为矢量不变,坐标系旋转。某些资料中的旋转为坐标系不变矢量旋转,在这种情况下式\ref{四元数旋转倒数表示}公式为$\bfxd{R} = Q \cdot \bfx{R} \cdot Q^{-1}$,对应的方向余弦阵也需要转置\cite{惯性技术}}下为$\bfxd{R}=x\bfid+y\bfjd{j}+z\bfkd{k}$,满足:
\begin{equation}\label{四元数旋转倒数表示}
    \bfxd{R}=Q^{-1} \cdot \bfx{R} \cdot Q
\end{equation} 

其中$Q$         表示旋转四元数,
$Q^{-1}$        表示旋转四元数的倒数.
四元数共厄定义:
\begin{equation}\label{四元数共厄}
    \begin{split}
        Q   & = q_0 + q_1\bfi + q_2\bfj + q_3\bfk \\
        Q^* & = q_0 - q_1\bfi - q_2\bfj - q_3\bfk
    \end{split}
\end{equation} 

而四元数归一化后$|Q|=1$,可以做如下推导:
\begin{equation*}
    \begin{split}
                    & Q^{-1} = \frac{1}{Q} = \frac{Q^*}{Q \cdot Q^*} = \frac{Q^*}{|Q|} = Q^* \\
        \Rightarrow & \bfxd{R}= Q^{-1} \cdot \bfx{R} \cdot Q = Q^* \cdot \bfx{R} \cdot Q
    \end{split}
\end{equation*}

即:
\begin{equation}\label{四元数旋转共厄表示}
    \bfxd{R}= Q^* \cdot \bfx{R} \cdot Q
\end{equation}

将式\ref{四元数旋转共厄表示}展开即:
\begin{equation}\label{四元数旋转展开}
    x'\bfi + y'\bfj + z'\bfk = (q_0 - q_1\bfi - q_2\bfj -q_3\bfk)(x\bfi + y\bfj + z\bfk)(q_0 + q_1\bfi + q_2\bfj + q_3\bfk)
\end{equation}

式\ref{四元数旋转展开}中为了美观同时不影响结果,将旋转前后的坐标轴的单位向量都表示为$\bfi$,$\bfj$,$\bfk$,
$q_x$           表示四元数的第$x$个分量,
$x'$,$y'$,$z'$  分别表示矢量在旋转后的坐标系下的分量值,
$x$,$y$, $z$    分别表示矢量参考坐标系下的分量值.
式\ref{四元数旋转展开}头两个括号使用分配率算出,共计12项.这12项组成一个括号,未参与计算的最后4项一个括号,然后每括号提一项拼尾巴可以化简得:
\begin{equation}\label{四元数旋转}
    \begin{split}
        \left[\begin{matrix}
                x' \\
                y' \\
                z'
        \end{matrix}\right]
        =
        \left[\begin{matrix}
                2(q_0^2+q_1^2)-1 & 2(q_1q_2+q_0q_3) & 2(q_1q_3-q_0q_2) \\
                2(q_1q_2-q_0q_3) & 2(q_0^2+q_2^2)-1 & 2(q_2q_3+q_0q_1) \\
                2(q_1q_3+q_0q_2) & 2(q_2q_3-q_0q_1) & 2(q_0^2+q_3^2)-1
        \end{matrix}\right]
        \left[\begin{matrix}
                x \\
                y \\
                z
        \end{matrix}\right]
    \end{split}
\end{equation}

\subsubsection{四元数转欧拉角}
结合式\ref{欧拉角转方向余弦阵},式\ref{四元数旋转}可以得到四元数转欧拉角的公式为:
\begin{equation}\label{四元数转欧拉角}
    \left\{\!\!\!\begin{array}{ll}
        \theta = \arctan2(q_2q_3+q_0q_1,q_0^2+q_3^2-0.5), & \theta \in (-\pi, \pi]\\
        \phi   =-\arcsin(2(q_1q_3-q_0q_2)), & \phi \in (-\pidtwo, \pidtwo] \\
        \psi   = \arctan2(q_1q_2+q_0q_3,q_0^2+q_1^2-0.5), & \psi \in (-\pi, \pi]
    \end{array}\right.
\end{equation}

式\ref{四元数转欧拉角}中俯仰角$\phi$虽然不足一个周期但是配合另外两个角度可以表示全姿态,至于欧拉角的万向锁问题笔者目前尚未理解透.

\subsection{四元数微积分}
四元数微分方程的推导笔者没看懂,如果想自己推导究可以研究参考文献\citet{四元数微分方程的推导},这里直接给出微分方程公式.
\begin{equation}\label{四元数微分方程}
    Q'(t) = \frac{1}{2}Q(t)\omega
\end{equation} 

其中$Q'(t)$ 表示当前时刻四元数微分,四元数是时间的函数
$Q(t)$      表示当强时刻四元数值,
$\omega$    表示s系下飞行器旋转角速度,即陀螺仪的输出.
为了简单,目前四元数积分算法采用一阶微分方程,如果后续实证精度不够再考虑高阶微分算法,求解四元数可以使用以下方程.
\begin{equation}\label{四元数积分方程}
    Q(t) = Q(t_0) + Q'(t_0) \cdot \delta t
\end{equation} 

其中$t$         表示当前时刻,是四元数函数的自变量,
$Q(t)$          表示当前时刻的四元数值,
$\delta t$      表示当前时刻与上次迭代时刻之间的间隔,
$Q(t_0)$ 表示本次迭代四元数初值,
$Q'(t_0)$表示上次迭代时刻的四元数微分.
式\ref{四元数积分方程}中除了$Q'(t_0)$都是已知量,$Q'(t_0)$的计算公式\ref{四元数微分方程}给出.所以四元数积分方程可以写成如下形式.
\begin{equation}\label{四元数递推方程}
    Q(t) = Q(t_0) + \frac{1}{2}Q(t_0)\omega \cdot \delta t
\end{equation} 

由于C语言中没有向量的概念,所以式\ref{四元数递推方程}需要写为分量形式\cite{飞行器专题资料}.
\begin{eqnarray}\label{四元数递推分量方程}
    \begin{split}
        \left[\begin{matrix}
                q_0(t) \\
                q_1(t) \\
                q_2(t) \\
                q_3(t)
        \end{matrix}\right]
        =
        \left[\begin{matrix}
                q_0(t_0) \\
                q_1(t_0) \\
                q_2(t_0) \\
                q_3(t_0)
        \end{matrix}\right]
        + 
        \frac{1}{2}
        \left[\begin{matrix}
                -\omega_x q_1(t_0)-\omega_y q_2(t_0)-\omega_z q_3(t_0) \\
                +\omega_x q_0(t_0)-\omega_y q_3(t_0)+\omega_z q_2(t_0) \\
                +\omega_x q_3(t_0)+\omega_y q_0(t_0)-\omega_z q_1(t_0) \\
                -\omega_x q_2(t_0)+\omega_y q_1(t_0)+\omega_z q_0(t_0) 
        \end{matrix}\right]
    \end{split}
\end{eqnarray} 

其中
$q_x(t)$        表示当前第$x$个四元数分量值,
$q_x(t_0)$      表示第$x$个四元数分量初始值,
$q_{x,t-1}$     表示上次迭代得到第$x$个姿态分量值,
$\omega_x$,$\omega_y$,$\omega_z$分别表示陀螺仪测量的$x$,$y$,$z$\footnote{分别为前右下}轴的角速度.

