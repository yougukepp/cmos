% section — subsection — subsubsection — paragraph — subparagraph

\section{四元数推导}
四元数表示姿态时仅需要4个变量,比方向余弦矩阵表示姿态\footnote{需要9个变量}少需要5个变量,同时也没有欧拉角的万向锁问题\footnote{我也不懂}.下面分析四元数的性质为算法设计打基础.

\subsection{定义}
四元数的定义如下:
\begin{equation}\label{四元数定义}
    Q = q_0 + q_1 \cdot \veci + q_2 \cdot \vecj + q_3 \cdot \veck
\end{equation} 

其中$Q$     表示四元数,
$q_x$       表示四元数的第$x$个分量,
$i$,$j$,$k$ 表示虚部.他们的关系为:
\begin{equation}\label{四元数虚部关系}
    \begin{split}
        \veci \cdot \veci &= \vecj \cdot \vecj = \veck \cdot \veck = -1 \\
        \veci \cdot \vecj &= -\vecj \cdot \veci = k \\
        \vecj \cdot \veck &= -\veck \cdot \vecj = i \\
        \veck \cdot \veci &= -\veci \cdot \veck = j
    \end{split}
\end{equation} 

单位四元数{归一化后}的定义如下:
\begin{equation}\label{单位四元数定义}
    Q = \costhetadtwo + \vec{n} \cdot \sinthetadtwo
\end{equation} 

其中$\vec{n}$   表示旋转轴向量,
$\theta$        表示旋转的角度.

\subsection{欧拉角转四元数}
123

