\subsection{基本概念}
基本概念从输入、输出、以及一些数学性质三个方面开始:
\subsubsection{输入}
姿态解算有5个已知量作为算法的输入,他们分别是:
\begin{itemize}
    \item 重力场

        与n系固连,恒定竖直向下,仅有天轴分量,是一个常矢量,与坐标系无关。
    \item 地磁场

        与n系固连,有一个北轴的分量和一个天轴的分量,但是没有东轴分量,是一个常矢量,与坐标系无关。
    \item 加速度计\footnote{加速度的工作原理可以参考加速度计原理\citep{加速度计原理}}数据

        该数据与s系固连,与重力场的方向相反\footnote{准确的说,加速度计的测量值叫比力加速度,表示作用在物体上非引力矢量和产生的加速度。如果飞行器受力不平衡会导致加速度计无法准确表征重力的反作用力的加速度,也就是说加速度计对干扰非常敏感,精确的姿态融合算法需要給加速度计滤波。},可部分\footnote{水平旋转角无法表征}表征姿态。
    \item 磁力计数据

        该数据与s系固连,指向地磁场的方向,可部分\footnote{绕地磁场轴的旋转无法表征}表征姿态。
    \item 陀螺仪数据

        与s系固连,表征s系旋转的角速度,可独立积分后估计姿态。
\end{itemize}

\subsubsection{输出}
姿态解算的输出有三个量,分别为\footnote{欧拉角与旋转顺序有关,本文使用航空次序角,也就是n系的$zyx$天东北}:
\begin{enumerate}
    \item 偏航角$\psi$

        绕天轴旋转的角度,用方向余弦阵表示\footnote{用极坐标很容易推导\citep{二维旋转}}: 
        \begin{eqnarray}\label{偏航角方向余弦}
            \begin{split}
                \mathrm{C}_z=\left[\begin{matrix}
                         \cos{\psi} & \sin{\psi} & 0 & \\
                        -\sin{\psi} & \cos{\psi} & 0 & \\
                                 0 &          0 & 1 &
                \end{matrix}\right]
            \end{split}
        \end{eqnarray}
    \item 俯仰角$\theta$

        绕东轴旋转的角度,用方向余弦阵表示: 
        \begin{eqnarray}\label{俯仰角方向余弦}
            \begin{split}
                \mathrm{C}_y=\left[\begin{matrix}
                        \cos{\theta} & 0 & -\sin{\theta} & \\
                                   0 & 1 &           0 & \\
                        \sin{\theta} & 0 &  \cos{\theta} &
                \end{matrix}\right]
            \end{split}
        \end{eqnarray}
    \item 横滚角$\phi$

        绕北轴旋转的角度,用方向余弦阵表示: 
        \begin{eqnarray}\label{横滚角方向余弦}
            \begin{split}
                \mathrm{C}_x=\left[\begin{matrix}
                        1 &           0 &          0 & \\
                        0 &  \cos{\phi} & \sin{\phi} & \\
                        0 & -\sin{\phi} & \cos{\phi} & \\
                \end{matrix}\right]
            \end{split}
        \end{eqnarray}
\end{enumerate}

利用$\psi,\theta,\phi$可以方便控制算法计算需要施加的控制量,完成姿态控制。

\subsubsection{四元数乘法展开}
四元数虚部单位运算法则:
\begin{eqnarray}\label{四元数虚部单位运算法则}
    \begin{split}
        &i^2=j^2=k^2=-1 \\
        &ij=-ji=k \\
        &jk=-kj=i \\
        &ki=-ik=j
    \end{split}
\end{eqnarray} 

注意\ref{四元数虚部单位运算法则}的顺序不可颠倒,四元数乘法展开\citep{捷联惯导航}: 
\begin{eqnarray}\label{四元数乘法展开}
    \begin{split}
        \bm{q}\cdot \bm{p}
        &=(a\quad ib\quad jc\quad kd)\cdot(e\quad if\quad jg\quad kh) \\
        &=\left[\begin{matrix}
        ae-bf-cg-dh \\
        af+be+ch-dg \\
        ag-bh+ce+df \\
        ah+bg-cf+de
        \end{matrix}\right]^\mathrm{T}
        \left[\begin{matrix}
                1 \\ i \\ j \\ k
        \end{matrix}\right] \\
        &=\left[\begin{matrix}
        a &-b &-c &-d \\
        b & a &-d & c \\
        c & d & a &-b \\
        d &-c & b & a
        \end{matrix}\right]
        \left[\begin{matrix} e \\ f \\ g \\ h \end{matrix}\right]
    \end{split}
\end{eqnarray} 

\subsubsection{四元数与欧拉角的转换}\label{section:四元数与欧拉角的转换}
方向余弦阵与旋转一一对应\citep{方向余弦阵},所以将四元数与欧拉角转换为对应的方向余弦阵可以得出它们相互转换的公式。n系到s系的方向余弦阵可以利用欧拉角\footnote{顺序为航空次序角$zyx$}\citep{捷联惯导航}求得:
\begin{equation}\label{欧拉角方向余弦}
    \begin{split} 
        \mathrm{C}^n_s=\mathrm{C}_x\mathrm{C}_y\mathrm{C}_z
    \end{split}
\end{equation} 

将式\ref{偏航角方向余弦},\ref{俯仰角方向余弦},\ref{横滚角方向余弦}带入\ref{欧拉角方向余弦}整理可得:
\begin{equation}\label{欧拉角n转s}
    \begin{split} 
        \mathrm{C}^n_s=&\left[\begin{matrix}
                1 &           0 &          0 & \\
                0 &  \cos{\phi} & \sin{\phi} & \\
                0 & -\sin{\phi} & \cos{\phi} & \\
                \end{matrix}\right]
                \left[\begin{matrix}
                \cos{\theta} & 0 & -\sin{\theta} & \\
                           0 & 1 &           0 & \\
                \sin{\theta} & 0 &  \cos{\theta} &
                \end{matrix}\right]
                \left[\begin{matrix}
                 \cos{\psi} & \sin{\psi} & 0 & \\
                -\sin{\psi} & \cos{\psi} & 0 & \\
                          0 &          0 & 1 &
                \end{matrix}\right] \\
                =&\left[\begin{matrix}
                \cos{\theta}\cos{\psi} & \cos{\theta}\sin{\psi} & -\sin{\psi} & \\
                \sin{\phi}\sin{\theta}\cos{\psi}-\cos{\phi}\sin{\psi} & \sin{\phi}\sin{\theta}\sin{\psi}+\cos{\phi}\cos{\psi} & \sin{\phi}\cos{\theta} & \\
                \cos{\phi}\sin{\theta}\cos{\psi}+\sin{\phi}\sin{\psi} & \cos{\phi}\sin{\theta}\sin{\psi}-\sin{\phi}\cos{\psi} & \cos{\phi}\cos{\theta} &
                \end{matrix}\right]
    \end{split}
\end{equation} 

由于n系到s系方向余弦矩阵与s系到n系的方向余弦阵互为转置\citep{方向余弦阵},所以s系到n系的转换公式为:
\begin{equation}\label{欧拉角s转n}
    \begin{split} 
        \mathrm{C}^s_n=&{\mathrm{C}^n_s}^\mathrm{T} \\
             =&\left[\begin{matrix}
                \cos{\theta}\cos{\psi} & \sin{\phi}\sin{\theta}\cos{\psi}-\cos{\phi}\sin{\psi} & \cos{\phi}\sin{\theta}\cos{\psi}+\sin{\phi}\sin{\psi} & \\
                \cos{\theta}\sin{\psi} & \sin{\phi}\sin{\theta}\sin{\psi}+\cos{\phi}\cos{\psi} & \cos{\phi}\sin{\theta}\sin{\psi}-\sin{\phi}\cos{\psi} & \\
                -\sin{\psi} & \sin{\phi}\cos{\theta} & \cos{\phi}\cos{\theta} &
                \end{matrix}\right]
    \end{split}
\end{equation} 

s系到n系的方向余弦阵可以利用单位四元数\citep{四元数矢量旋转证明1,四元数矢量旋转证明2}求得:
\begin{equation}\label{四元数旋转向量}
    \begin{split} 
        ^n\bm{r} 
        = &^s_n\bm{q} ^s\bm{r} ^s_n\bm{q}^{-1} \\
        = &^s_n\bm{q} ^s\bm{r} ^s_n\bm{q}^{*} \\
        = &\mathrm{C}^s_n \cdot ^s\bm{r}
    \end{split}
\end{equation} 

$^s\bm{r}$          表示矢量$\bm{r}$在s系下的坐标;
$^n\bm{r}$          表示矢量$\bm{r}$在n系下的坐标;
$^s_n\bm{q}$        表示s系到n系的旋转四元数;
$^s_n\bm{q}^{-1}$   表示四元数$^s_n\bm{q}$的逆\citep{四元数矢量旋转证明1,四元数矢量旋转证明2};
$^s_n\bm{q}^{*}$    表示四元数$^s_n\bm{q}$的共厄\citep{四元数矢量旋转证明1,四元数矢量旋转证明2};
$\mathrm{C}^s_n$    表示s系到n系的方向余弦。

连续应用四元数乘法展开式\ref{四元数乘法展开}化简式\ref{四元数旋转向量}得:
\begin{equation}\label{四元数方向余弦阵}
    \begin{split} 
        C^s_n = \left[\begin{matrix} 
                (q^2_0+q^2_1-q^2_2-q^2_3) & (2q_1q_2-2q_0q_3) & (2q_1q_3+2q_0q_2) & \\
                (2q_1q_2+2q_0q_3) & (q^2_0-q^2_1+q^2_2-q^2_3) & (2q_2q_3-2q_0q_1) & \\
                (2q_1q_3-2q_0q_2) & (2q_2q_3+2q_0q_1) & (q^2_0-q^2_1-q^2_2+q^2_3) &
        \end{matrix}\right]
    \end{split}
\end{equation} 

方向余弦写为分量形式:
\begin{equation}\label{方向余弦}
    \begin{split} 
        C^s_n = \left[\begin{matrix} 
                C_{1,1} & C_{1,2} & C_{1,3} & \\
                C_{2,1} & C_{2,2} & C_{2,3} & \\
                C_{3,1} & C_{3,2} & C_{3,3} &
        \end{matrix}\right]
    \end{split}
\end{equation} 

结合式\ref{欧拉角s转n}、式\ref{四元数方向余弦阵}和式\ref{方向余弦}得\footnote{推导过于繁琐,分别从\citep{捷联惯导航},\citep{9轴融合论文}中抄录},需要重推,有问题:
\begin{equation}\label{欧拉角转四元数}
    \begin{split} 
        q_0=\cos\frac{\phi}{2}\cos\frac{\theta}{2}\cos\frac{\psi}{2}+\sin\frac{\phi}{2}\sin\frac{\theta}{2}\sin\frac{\psi}{2} \\
        q_1=\sin\frac{\phi}{2}\cos\frac{\theta}{2}\cos\frac{\psi}{2}-\cos\frac{\phi}{2}\sin\frac{\theta}{2}\sin\frac{\psi}{2} \\
        q_2=\cos\frac{\phi}{2}\sin\frac{\theta}{2}\cos\frac{\psi}{2}+\sin\frac{\phi}{2}\cos\frac{\theta}{2}\sin\frac{\psi}{2} \\
        q_3=\cos\frac{\phi}{2}\cos\frac{\theta}{2}\sin\frac{\psi}{2}+\sin\frac{\phi}{2}\sin\frac{\theta}{2}\cos\frac{\psi}{2}
    \end{split}
\end{equation} 
\begin{equation}\label{四元数转欧拉角}
    \begin{split} 
        \psi=&Aatan2(2q_1q_2-2q_0q_3,2q^2_0+2q^2_1-1) \\
        \theta=&-\arcsin(2q_1q_3+2q_0q_2) \\
        \phi=&Atan2(2q_2q_3-2q_0q_1,2q^2_0+2q^2_3-1)
    \end{split}
\end{equation} 
式\ref{欧拉角转四元数}是欧拉角转四元数公式,式\ref{四元数转欧拉角}是四元数转欧拉角公式。

\subsubsection{四元数归一化}
四元数归一化\citep{四元数矢量旋转证明1,四元数矢量旋转证明2}后才能方便的表示旋转\footnote{应用式\ref{四元数旋转向量}},下面对四元数进行归一化:
\begin{equation}\label{四元数归一化}
    \begin{split}
        |\bm{q}|&=\bm{q} \cdot \bm{q}^*={q_0^2+q_1^2+q_2^2+q_3^2}\\ 
        ^s_n\bm{q}_{norm}&=\frac{^s_n\bm{q}}{|^s_n\bm{q}|}
        =\left[\frac{q_{0,t}}{|^s_n\bm{q}|}\quad
            \frac{q_{1,t}}{|^s_n\bm{q}|}\quad
            \frac{q_{2,t}}{|^s_n\bm{q}|}\quad
             \frac{q_{3,t}}{|^s_n\bm{q}|}\right]^\mathrm{T}
    \end{split}
\end{equation} 

\subsubsection{小结}
姿态融合算法主要分为三个部分实现:
\begin{enumerate}
    \item 3轴姿态

        使用陀螺仪的数据和四元数微分方程 用积分法获取估计姿态
    \item 6轴融合

        使用陀螺仪、加速度计的数据和梯度下降算法
    \item 9轴融合

        使用陀螺仪、加速度计、磁力计的数据和梯度下降算法
\end{enumerate}

