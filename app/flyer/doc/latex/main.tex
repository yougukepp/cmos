\documentclass[10pt,a4paper]{article}

% 水印
\usepackage{draftwatermark}
\SetWatermarkScale{1}
%\SetWatermarkColor[rgb]{1,0,0}
\SetWatermarkLightness{0.9}
\SetWatermarkText{初稿}

% 加粗希腊字母
\usepackage{bm}
% 超链接
\usepackage[colorlinks,linkcolor=red]{hyperref}

% 中文
\usepackage{xeCJK}
\setCJKmainfont{文泉驿等宽正黑}
%\setCJKmainfont[AutoFakeBold=true]{文泉驿等宽微米黑}
% 断行
\XeTeXlinebreaklocale "zh"
\parindent 2em
\usepackage{indentfirst}
% 将日期变为中文格式
\renewcommand{\today}{\number\year 年 \number\month 月 \number\day 日}
% 将目录标题改为中文
\renewcommand{\contentsname}{\centerline{目录}}
\renewcommand{\abstractname}{摘要}
\renewcommand{\figurename}{图}
\renewcommand{\tablename}{表}

% 公式
\usepackage{amsmath}
\usepackage{amssymb}
% 公式编号
\makeatletter
\@addtoreset{equation}{section}
\makeatother 
\renewcommand\theequation{\oldstylenums{\thesection} .\oldstylenums{\arabic{equation}}}
% 公式快捷命令
%\newcommand{name}[num]{definition}
\newcommand{\bfx}[1]{\mathbf{#1}}
\newcommand{\bfxd}[1]{\mathbf{#1}'}
\newcommand{\bfi}{\bfx{i}}
\newcommand{\bfj}{\bfx{j}}
\newcommand{\bfk}{\bfx{k}}
\newcommand{\bfid}{\bfxd{i}}
\newcommand{\bfjd}{\bfxd{j}}
\newcommand{\bfkd}{\bfxd{k}}

\newcommand{\sinphi}{\sin{\phi}}
\newcommand{\cosphi}{\cos{\phi}}
\newcommand{\sintheta}{\sin{\theta}}
\newcommand{\costheta}{\cos{\theta}}
\newcommand{\sinpsi}{\sin{\psi}}
\newcommand{\cospsi}{\cos{\psi}}
\newcommand{\sinphidtwo}{\sin\frac{\phi}{2}}
\newcommand{\cosphidtwo}{\cos\frac{\phi}{2}}
\newcommand{\sinthetadtwo}{\sin\frac{\theta}{2}}
\newcommand{\costhetadtwo}{\cos\frac{\theta}{2}}
\newcommand{\sinpsidtwo}{\sin\frac{\psi}{2}}
\newcommand{\cospsidtwo}{\cos\frac{\psi}{2}}

\newcommand{\pidtwo}{\frac{\pi}{2}}
\newcommand{\transpose}{^\mathbf{T}}

% 参考文献
\usepackage[numbers,sort&compress]{natbib}
\renewcommand{\citet}[1]{\textsuperscript{\cite{#1}}}
\renewcommand{\citep}[1]{\textsuperscript{\cite{#1}}}
\addtolength{\bibsep}{-0.5 em} % 缩小参考文献间的垂直间距
\setlength{\bibhang}{2em}
\newcommand{\bibnumfont}[1]{\textit{#1}}

% 制作标题目录
\usepackage{makeidx}
\makeindex
\printindex
\setcounter{secnumdepth}{3}
\title{姿态融合算法}
%\author{彭鹏(QQ:516190948)}
\author{彭鹏}

\begin{document}
\maketitle
\newpage
\tableofcontents
%\listoffigures
%\listoftables
\newpage

\begin{abstract}
四轴飞行器主要包括:姿态算法和控制算法两种算法。姿态算法完成飞行器自身姿态的计算,控制算法使用姿态计算出需要施加的控制量完成姿态控制,本文主要分析姿态算法。

姿态算法是四轴飞行器的核心算法,姿态算法的核心器件是陀螺仪,陀螺仪可以测量自身旋转的角速度。理论上只要陀螺仪精度足够,结合初始姿态通过积分可以计算飞行器任意时刻的姿态。然并卵,现实是残酷的,由于成本和尺寸的限制,四轴只能使用MEMS陀螺仪。MEMS陀螺仪精度不佳并有较大的漂移,在长时间的积分过程中会导致很大的积分误差,导致估计姿态随着时间累计而变得不可接受\citep{捷联惯导航}。

怎么办呢?\citep{9轴融合论文}联想轮船,陀螺仪类似舵手,舵手长时间独立工作后会将船带偏,这个时候需要了望手来不断纠偏。通常四轴中使用加速度计来充当了望手,纠偏轮船的方向。加速度计通过测量重力场的方向纠正陀螺仪积分得到的姿态,普通的四轴飞行器使用陀螺仪和加速度计就可以完成精度可接受的姿态解算。但是有一个问题,重力场方向竖直向下,可以纠正飞行器的水平倾斜,但飞行器绕重力场方向的旋转没法纠正。所以仅用6轴(陀螺仪3轴,加速度计3轴)融合的姿态只能保证飞行器的水平平衡,无法保证其航向稳定\footnote{会水平旋转},说通俗一点也就是飞行器会找不到北。类似加速度计的思路,传感器组合中加入一个测量地磁的传感器\citep{经典博客},就可以纠正飞行器航向上的偏移,这就是通常所说的9轴姿态融合。
\end{abstract}
\newpage

% section — subsection — subsubsection — paragraph — subparagraph
\section{术语符号说明} 
以下是术语符号表:
\begin{table}[!hbp]
\begin{center}
    \begin{tabular}{|l|l|}
        \hline
        术语名 & 含义 \\
        \hline
        n系 & 导航坐标系,它是恒定的参考坐标系.$x$,$y$,$z$分别表示地理北东地\footnote{方便起见,采用右手坐标系} \\
        \hline
        s系 & 固定与飞行器的坐标系, 随飞行器一同旋转,$x$,$y$,$z$分别表示飞行器前右下\footnote{方便起见,采用右手坐标系} \\
        \hline
        姿态 & 飞行器的旋转,可以用s系想对于n系的旋转替换 \\
        \hline
        四元数 & 陀螺仪算法中飞行器姿态的数学表示 \\
        \hline
        欧拉角 & 用来确定飞行器姿态的3个旋转角度,本文使用$Z$-$Y$-$X$顺序 \\
        \hline
        估计姿态 & 陀螺仪积分算出的姿态 \\
        \hline
        直接姿态 & 加计和磁计直接测出的姿态 \\
        \hline
        比力加速度 & 表示作用在物体上非引力矢量和产生的加速度,加计实际测量量 \\
        \hline
        指东针 & 重力矢量叉乘地磁矢量的到的结果矢量,该矢量指向n系的正东 \\
        \hline
    \end{tabular}
    \caption{术语表\label{术语表}}
\end{center}
\end{table}

文中的陀螺仪、加速度计、磁力计都是三轴的。下面先对姿态算法的用到的基本原理作出说明,然后根据原理总结出姿态算法。如果只想使用算法仅仅看结论\ref{section:最终结论}即可。
\newpage

\section{基本原理}
陀螺仪测量旋转角速度,通过对角速度积分可以获取姿态.

加计\footnote{后文加速度计简称加计}测试的物理量为比力.当飞行器受力平衡时,飞行器比力与重力方向相反.在有干扰情况下比力值会出现较大幅度突变,通过滤波算法可以抑制干扰的影响,后文尽在设计加计滤波算法是考虑干扰,其他情况下认为飞行器受力平衡.在飞行器工作过程中,采样加计的输出并将它转换到n系,比较他与重力$\mathbf{g}=(0,0,1)$可以求得水平方向的两欧拉角分量:横滚角,俯仰角.使用这两个角度纠正陀螺仪积分得出的对应值完成水平姿态融合.

磁计\footnote{后问磁力计简称磁计}测量的物理量为地磁场.由于地磁场并非正北方向,导致不便,本文使用重力场叉乘地磁场构造出一个正东方向的"指东针",该向量仅有东向分量.便于计算,代价是多两次叉乘.至于是否划算暂不考虑,毕竟过早优化才是万恶之源.有了指东针,可以结合加计与磁计的输出叉乘出"指东针"测量值,将该测量转换到n系,比较他与实际"指东针"$\mathbf{e}=(0,1,0)$可以求得最后一个欧拉角分量偏航角,使用它纠正陀螺仪积分得出的偏航角完成偏航角融合。至次,完成了飞行器的全姿态融合.

在三个传感器中陀螺仪器是和行器件,他持续工作积分跟踪姿态,加计和磁计是辅助器件,用于校正陀螺仪的误差.

为了便于分析,假定飞行器启动前,s系前右下与n系北东地分别重合\footnote{飞行器水平正放,机头指向正北},飞行器的姿态解算转换为求s系相对于n系的旋转角. 本文在分析陀螺积分算法时为了减少运算量使用四元数表示姿态,分析直接姿态算法时出于简单\footnote{目前笔者能自己想明白的只有欧拉角算法,其他的互补滤波,卡尔曼滤波搞懂了再补上.}考虑使用欧拉角表示姿态,推导过程中采用方向余弦分析分析欧拉角与四元数的转换,出于方便姿态解算的输出使用欧拉角.

\section{欧拉角与方向余弦}
首先分析二维位旋转.
\begin{figure}[!hbp]
    \begin{center}
        \includegraphics[height=5cm, width=5cm]{fig/2dxoyrotate.pdf}
        \caption{二维旋转}\label{二维旋转}
    \end{center}
\end{figure}

这里有一点需要强调,本文坐标系统一采用右手坐标系,结合绘图的美观性,所有旋转轴\footnote{旋转所绕的轴,而非旋转的轴}指向读者.


\section{四元数推导}
123

% section — subsection — subsubsection — paragraph — subparagraph

\section{算法步骤}\label{section:最终结论}
下面总结3轴融合,6轴融合,9轴融合的步骤。
\subsection{3轴融合}
3轴融合算法可以分解为一下步骤:
\begin{enumerate}
    \item 四元数初始化

        积分操作需要初始值,而四元数的几何意义不是很明确,难以确定其初始值.故四元数的初始值需要通过欧拉角计算.初始欧拉角全为零,代入式\ref{欧拉角转四元数},可以得到四元数的初值.
    \item 微积分求四元数

        通过陀螺仪输出的角速度求四元数微分,利用该微分值求四元数.四元数微积分公式为式\ref{四元数递推分量方程}.
    \item 计算瞬时欧拉角

        为了和加计,磁计融合以及便于后面的姿态控制,四元数需要转换为有物理含义的欧拉角.四元数转欧拉角使用公式\ref{四元数转欧拉角}.
\end{enumerate}

\subsection{6轴融合}
6轴融合

\subsection{9轴融合}
9轴融合




\section{后续工作}
\begin{enumerate}
    \item 卡尔曼滤波

        温度计和房间温度的例子
    \item 加速度标定

        取平均或最小二乘法
    \item 控制算法

        PID算法
\end{enumerate}

% 参考文献
\newpage
% 将参考文献标题改为中文
% article 使用
\renewcommand\refname{参考文献}
% book 使用
%\renewcommand\bibname{参考文献}
\centering %居中
\bibliographystyle{plain}
\bibliography{main}

\end{document}

