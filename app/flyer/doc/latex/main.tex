\documentclass[10pt,a4paper]{article}

% 加粗希腊字母
\usepackage{bm}
% 超链接
\usepackage[colorlinks,linkcolor=red]{hyperref}

% 中文
\usepackage{xeCJK}
\setCJKmainfont{文泉驿等宽微米黑:style=Regular}
% 断行
\XeTeXlinebreaklocale "zh"
\parindent 2em
\usepackage{indentfirst}
% 将日期变为中文格式
\renewcommand{\today}{\number\year 年 \number\month 月 \number\day 日}
% 将目录标题改为中文
\renewcommand{\contentsname}{目录}
\renewcommand{\abstractname}{摘要}

% 公式
\usepackage{amsmath}
% 公式编号
\makeatletter
\@addtoreset{equation}{section}
\makeatother 
\renewcommand\theequation{\oldstylenums{\thesection} .\oldstylenums{\arabic{equation}}}
% 公式快捷命令

% 参考文献
\usepackage[numbers,sort&compress]{natbib}
\renewcommand{\citet}[1]{\textsuperscript{\cite{#1}}}
\renewcommand{\citep}[1]{\textsuperscript{\cite{#1}}}
\addtolength{\bibsep}{-0.5 em} % 缩小参考文献间的垂直间距
\setlength{\bibhang}{2em}
\newcommand{\bibnumfont}[1]{\textit{#1}}

% 制作标题目录
\usepackage{makeidx}
\makeindex
\printindex
\setcounter{secnumdepth}{5}
\title{姿态融合算法}
\author{彭鹏(QQ:516190948)}

\begin{document}
\maketitle
\newpage
\tableofcontents
%\listoffigures
%\listoftables
\newpage

\begin{abstract}
四轴飞行器主要包括:姿态算法和控制算法两种算法。姿态算法完成飞行器自身姿态的计算,控制算法使用姿态计算出需要施加的控制量完成姿态控制,本文主要分析姿态算法。

姿态算法是四轴飞行器的核心算法,姿态算法的核心器件是陀螺仪,陀螺仪可以测量自身旋转的角速度。理论上只要陀螺仪精度足够,结合初始姿态通过积分可以计算飞行器任意时刻的姿态。然并卵,现实是残酷的,由于成本和尺寸的限制,四轴只能使用MEMS陀螺仪。MEMS陀螺仪精度不佳并有较大的漂移,在长时间的积分过程中会导致很大的积分误差,导致估计姿态随着时间累计而变得不可接受\citep{捷联惯导航}。

怎么办呢?\citep{9轴融合论文}联想轮船,陀螺仪类似舵手,舵手长时间独立工作后会将船带偏,这个时候需要了望手来不断纠偏。通常四轴中使用加速度计来充当了望手,纠偏轮船的方向。加速度计通过测量重力场的方向纠正陀螺仪积分得到的姿态,普通的四轴飞行器使用陀螺仪和加速度计就可以完成精度可接受的姿态解算。但是有一个问题,重力场方向竖直向下,可以纠正飞行器的水平倾斜,但飞行器绕重力场方向的旋转没法纠正。所以仅用6轴(陀螺仪3轴,加速度计3轴)融合的姿态只能保证飞行器的水平平衡,无法保证其航向稳定\footnote{会水平旋转},说通俗一点也就是飞行器会找不到北。类似加速度计的思路,传感器组合中加入一个测量地磁的传感器\citep{经典博客},就可以纠正飞行器航向上的偏移,这就是通常所说的9轴姿态融合。
\end{abstract}
\newpage

% section — subsection — subsubsection — paragraph — subparagraph
\section{术语符号说明} 
以下是术语符号表:
\begin{table}[!hbp]
\begin{center}
    \begin{tabular}{|l|l|}
        \hline
        术语名 & 含义 \\
        \hline
        n系 & 导航坐标系,它是恒定的参考坐标系.$x$,$y$,$z$分别表示地理北东地\footnote{方便起见,采用右手坐标系} \\
        \hline
        s系 & 固定与飞行器的坐标系, 随飞行器一同旋转,$x$,$y$,$z$分别表示飞行器前右下\footnote{方便起见,采用右手坐标系} \\
        \hline
        姿态 & 飞行器的旋转,可以用s系想对于n系的旋转替换 \\
        \hline
        四元数 & 陀螺仪算法中飞行器姿态的数学表示 \\
        \hline
        欧拉角 & 用来确定飞行器姿态的3个旋转角度,本文使用$Z$-$Y$-$X$顺序 \\
        \hline
        估计姿态 & 陀螺仪积分算出的姿态 \\
        \hline
        直接姿态 & 加计和磁计直接测出的姿态 \\
        \hline
        比力加速度 & 表示作用在物体上非引力矢量和产生的加速度,加计实际测量量 \\
        \hline
        指东针 & 重力矢量叉乘地磁矢量的到的结果矢量,该矢量指向n系的正东 \\
        \hline
    \end{tabular}
    \caption{术语表\label{术语表}}
\end{center}
\end{table}

文中的陀螺仪、加速度计、磁力计都是三轴的。下面先对姿态算法的用到的基本原理作出说明,然后根据原理总结出姿态算法。如果只想使用算法仅仅看结论\ref{section:最终结论}即可。
\newpage

\section{基本原理}
陀螺仪测量旋转角速度,通过对角速度积分可以获取姿态.

加计\footnote{后文加速度计简称加计}测试的物理量为比力.当飞行器受力平衡时,飞行器比力与重力方向相反.在有干扰情况下比力值会出现较大幅度突变,通过滤波算法可以抑制干扰的影响,后文尽在设计加计滤波算法是考虑干扰,其他情况下认为飞行器受力平衡.在飞行器工作过程中,采样加计的输出并将它转换到n系,比较他与重力$\vec{g}=(0,0,1)$可以求得水平方向的两欧拉角分量:横滚角,俯仰角.使用这两个角度纠正陀螺仪积分得出的对应值完成水平姿态融合.

磁计\footnote{后问磁力计简称磁计}测量的物理量为地磁场.由于地磁场并非正北方向,导致不便,本文使用重力场叉乘地磁场构造出一个正东方向的"指东针",该向量仅有东向分量.便于计算,代价是多两次叉乘.至于是否划算暂不考虑,毕竟过早优化才是万恶之源.有了指东针,可以结合加计与磁计的输出叉乘出"指东针"测量值,将该测量转换到n系,比较他与实际"指东针"$\vec{e}=(0,1,0)$可以求得最后一个欧拉角分量偏航角,使用它纠正陀螺仪积分得出的偏航角完成偏航角融合。至次,完成了飞行器的全姿态融合.

为了便于分析,假定飞行器启动前,s系前右下与n系北东地分别重合,飞行器的姿态解算转换为求s系相对于n系的旋转角. 本文在分析陀螺积分算法时为了减少运算量使用四元数表示姿态,分析直接姿态算法时出于简单\footnote{目前笔者能自己想明白的只有欧拉角算法,其他的互补滤波,卡尔曼滤波搞懂了再补上.}考虑使用欧拉角表示姿态,推导过程中采用方向余弦分析分析欧拉角与四元数的转换,出于方便姿态解算的输出使用欧拉角.
% section — subsection — subsubsection — paragraph — subparagraph

\section{四元数推导}
四元数表示姿态时仅需要4个变量,比方向余弦矩阵表示姿态\footnote{需要9个变量}少需要5个变量,同时也没有欧拉角的万向锁问题\footnote{我也不懂}.下面分析四元数的性质为算法设计打基础.

\subsection{定义}
四元数的定义如下:
\begin{equation}\label{四元数定义}
    Q = q_0 + q_1\bfi + q_2\bfj + q_3\bfk
\end{equation} 

其中$Q$     表示四元数,
$q_x$       表示四元数的第$x$个分量,
$\bfi$,$\bfj$,$bfk$表示虚部.他们的关系为:
\begin{equation}\label{四元数虚部关系}
    \begin{split}
        & \bfi \cdot \bfi =  \bfj \cdot \bfj = \bfk \cdot \bfk = -1 \\
        & \bfi \cdot \bfj = -\bfj \cdot \bfi = \bfk \\
        & \bfj \cdot \bfk = -\bfk \cdot \bfj = \bfi \\
        & \bfk \cdot \bfi = -\bfi \cdot \bfk = \bfj
    \end{split}
\end{equation} 

单位四元数{归一化后}的定义如下:
\begin{equation}\label{四元数归一化定义}
    \begin{split}
        Q = & \costhetadtwo + \bfx{n} \cdot \sinthetadtwo \\
        \bfx{n} = & \cos{\alpha}i + \cos{\beta}j + \cos{\gamma}k
    \end{split}
\end{equation} 

其中$\theta$    表示旋转的角度,
$\bfx{n}$       表示旋转轴向量,
$\alpha$        表示旋转轴与参考系$x$轴夹角,
$\beta$         表示旋转轴与参考系$y$轴夹角,
$\gamma$        表示旋转轴与参考系$z$轴夹角.

\subsection{欧拉角转四元数}
本文采用的欧拉角顺序为$Z$-$Y$-$X$,所以可以用三次旋转表示\cite{飞行器专题资料}旋转\footnote{s系下推导}.
\begin{equation}\label{四元数三次旋转}
    Q_{xyz} = Q_z \cdot Q_y \cdot Q_y = (\cospsidtwo + \sinpsidtwo \bfk) \cdot (\cosphidtwo + \sinphidtwo \bfj) \cdot (\costhetadtwo + \sinthetadtwo \bfk)
\end{equation} 

其中$Q_{xyz}$   表示旋转四元数,
$Q_{z}$         表示绕$z$轴的旋转四元数,
$Q_{y}$         表示绕$y$轴的旋转四元数,
$Q_{x}$         表示绕$x$轴的旋转四元数,
$\theta$        表示横滚角,
$\phi$          表示俯仰角,
$\psi$          表示偏航角.将式\ref{四元数三次旋转}展开\footnote{每个括号提一项拼尾巴$1$,$\bfi$,$\bfj$,$\bfk$}可以得到:
\begin{equation}\label{欧拉角转四元数}
    \begin{split}
        \left[\begin{matrix}
                q_0 \\
                q_1 \\
                q_2 \\
                q_3
        \end{matrix}\right]\transpose
        =
        \left[\begin{matrix}
                \costhetadtwo\cosphidtwo\cospsidtwo + \sinthetadtwo\sinphidtwo\sinpsidtwo \\
                \sinthetadtwo\cosphidtwo\cospsidtwo - \costhetadtwo\sinphidtwo\sinpsidtwo \\
                \costhetadtwo\sinphidtwo\cospsidtwo + \sinthetadtwo\cosphidtwo\sinpsidtwo \\
                \costhetadtwo\cosphidtwo\sinpsidtwo - \sinthetadtwo\sinphidtwo\cospsidtwo
        \end{matrix}\right]\transpose
        \left[\begin{matrix}
                1 \\
                \bfi \\
                \bfj \\
                \bfk \\
        \end{matrix}\right]
    \end{split}
\end{equation} 

其中$\theta$    表示横滚角,
$\phi$          表示俯仰角,
$\psi$          表示偏航角,
$q_x$           表示四元数的第$x$个分量.式\ref{欧拉角转四元数}是欧拉角转换为四元数的公式.

\subsection{四元数转欧拉角}
由于笔者尚未发现四元数直接转换欧拉角的推导,所以这里先推导四元数转方向余弦阵,再利用方向余弦阵推到四元数转欧拉角的公式.
\subsubsection{四元数转方向余弦阵}
设矢量$\bfx{R}=x\bfi+y\bfj+z\bfk$,在\textcolor[rgb]{1,0,0}{旋转后的坐标系}\footnote{本文的旋转在未说明的情况下为矢量不变,坐标系旋转。某些资料中的旋转为坐标系不变矢量旋转,在这种情况下式\ref{四元数旋转倒数表示}公式为$\bfxd{R} = Q \cdot \bfx{R} \cdot Q^{-1}$,对应的方向余弦阵也需要转置\cite{惯性技术}}下为$\bfxd{R}=x\bfid+y\bfjd{j}+z\bfkd{k}$,满足:
\begin{equation}\label{四元数旋转倒数表示}
    \bfxd{R}=Q^{-1} \cdot \bfx{R} \cdot Q
\end{equation} 

其中$Q$         表示旋转四元数,
$Q^{-1}$        表示旋转四元数的倒数.
四元数共厄定义:
\begin{equation}\label{四元数共厄}
    \begin{split}
        Q   & = q_0 + q_1\bfi + q_2\bfj + q_3\bfk \\
        Q^* & = q_0 - q_1\bfi - q_2\bfj - q_3\bfk
    \end{split}
\end{equation} 

而四元数归一化后$|Q|=1$,可以做如下推导:
\begin{equation*}
    \begin{split}
                    & Q^{-1} = \frac{1}{Q} = \frac{Q^*}{Q \cdot Q^*} = \frac{Q^*}{|Q|} = Q^* \\
        \Rightarrow & \bfxd{R}= Q^{-1} \cdot \bfx{R} \cdot Q = Q^* \cdot \bfx{R} \cdot Q
    \end{split}
\end{equation*}

即:
\begin{equation}\label{四元数旋转共厄表示}
    \bfxd{R}= Q^* \cdot \bfx{R} \cdot Q
\end{equation}

将式\ref{四元数旋转共厄表示}展开即:
\begin{equation}\label{四元数旋转展开}
    x'\bfi + y'\bfj + z'\bfk = (q_0 - q_1\bfi - q_2\bfj -q_3\bfk)(x\bfi + y\bfj + z\bfk)(q_0 + q_1\bfi + q_2\bfj + q_3\bfk)
\end{equation}

式\ref{四元数旋转展开}中为了美观同时不影响结果,将旋转前后的坐标轴的单位向量都表示为$\bfi$,$\bfj$,$\bfk$,
$q_x$           表示四元数的第$x$个分量,
$x'$,$y'$,$z'$  分别表示矢量在旋转后的坐标系下的分量值,
$x$,$y$, $z$    分别表示矢量参考坐标系下的分量值.
式\ref{四元数旋转展开}头两个括号使用分配率算出,共计12项.这12项组成一个括号,未参与计算的最后4项一个括号,然后每括号提一项拼尾巴可以化简得:
\begin{equation}\label{四元数旋转}
    \begin{split}
        \left[\begin{matrix}
                x' \\
                y' \\
                z'
        \end{matrix}\right]
        =
        \left[\begin{matrix}
                2(q_0^2+q_1^2)-1 & 2(q_1q_2+q_0q_3) & 2(q_1q_3-q_0q_2) \\
                2(q_1q_2-q_0q_3) & 2(q_0^2+q_2^2)-1 & 2(q_2q_3+q_0q_1) \\
                2(q_1q_3+q_0q_2) & 2(q_2q_3-q_0q_1) & 2(q_0^2+q_3^2)-1
        \end{matrix}\right]
        \left[\begin{matrix}
                x \\
                y \\
                z
        \end{matrix}\right]
    \end{split}
\end{equation}

\subsubsection{四元数转欧拉角}
结合式\ref{方向余弦阵定义},式\ref{四元数旋转}可以得到四元数转欧拉角的公式为:
\begin{equation}\label{四元数转欧拉角}
    \left\{\!\!\!\begin{array}{ll}
        \theta = \arctan2(q_2q_3+q_0q_1,q_0^2+q_3^2-0.5), & \theta \in (-\pi, \pi]\\
        \phi   =-\arcsin(2(q_1q_3-q_0q_2)), & \phi \in (-\pidtwo, \pidtwo] \\
        \psi   = \arctan2(q_1q_2+q_0q_3,q_0^2+q_1^2-0.5), & \psi \in (-\pi, \pi]
    \end{array}\right.
\end{equation}

式\ref{四元数转欧拉角}中俯仰角$\phi$虽然不足一个周期但是配合另外两个角度可以表示全姿态,至于欧拉角的万向锁问题笔者目前尚未理解透.

\subsection{四元数微积分}
四元数微分方程的推导笔者没看懂,如果想自己推导究可以研究参考文献\citet{四元数微分方程的推导},这里直接给出微分方程公式.
\begin{equation}\label{四元数微分方程}
    Q'(t) = \frac{1}{2}Q(t)\omega
\end{equation} 

其中$Q'(t)$ 表示当前时刻四元数微分,四元数是时间的函数
$Q(t)$      表示当强时刻四元数值,
$\omega$    表示s系下飞行器旋转角速度,即陀螺仪的输出.
为了简单,目前四元数积分算法采用一阶微分方程,如果后续实证精度不够再考虑高阶微分算法,求解四元数可以使用以下方程.
\begin{equation}\label{四元数积分方程}
    Q(t) = Q(t_0) + Q'(t_0) \cdot \delta t
\end{equation} 

其中$t$         表示当前时刻,是四元数函数的自变量,
$Q(t)$          表示当前时刻的四元数值,
$\delta t$      表示当前时刻与上次迭代时刻之间的间隔,
$Q(t_0)$ 表示本次迭代四元数初值,
$Q'(t_0)$表示上次迭代时刻的四元数微分.
式\ref{四元数积分方程}中除了$Q'(t_0)$都是已知量,$Q'(t_0)$的计算公式\ref{四元数微分方程}给出.所以四元数积分方程可以写成如下形式.
\begin{equation}\label{四元数递推方程}
    Q(t) = Q(t_0) + \frac{1}{2}Q(t_0)\omega \cdot \delta t
\end{equation} 

由于C语言中没有向量的概念,所以式\ref{四元数递推方程}需要写为分量形式\cite{飞行器专题资料}.
\begin{eqnarray}\label{四元数递推分量方程}
    \begin{split}
        \left[\begin{matrix}
                q_0(t) \\
                q_1(t) \\
                q_2(t) \\
                q_3(t)
        \end{matrix}\right]
        =
        \left[\begin{matrix}
                q_0(t_0) \\
                q_1(t_0) \\
                q_2(t_0) \\
                q_3(t_0)
        \end{matrix}\right]
        + 
        \frac{1}{2}
        \left[\begin{matrix}
                -\omega_x q_1(t_0)-\omega_y q_2(t_0)-\omega_z q_3(t_0) \\
                +\omega_x q_0(t_0)-\omega_y q_3(t_0)+\omega_z q_2(t_0) \\
                +\omega_x q_3(t_0)+\omega_y q_0(t_0)-\omega_z q_1(t_0) \\
                -\omega_x q_2(t_0)+\omega_y q_1(t_0)+\omega_z q_0(t_0) 
        \end{matrix}\right]
    \end{split}
\end{eqnarray} 

其中
$q_x(t)$        表示当前第$x$个四元数分量值,
$q_x(t_0)$      表示第$x$个四元数分量初始值,
$q_{x,t-1}$     表示上次迭代得到第$x$个姿态分量值,
$\omega_x$,$\omega_y$,$\omega_z$分别表示陀螺仪测量的$x$,$y$,$z$\footnote{分别为前右下}轴的角速度.



\section{算法步骤}\label{section:最终结论}
下面总结3轴融合,6轴融合,9轴融合的步骤。
\subsection{3轴融合}
3轴融合

\subsection{6轴融合}
6轴融合

\subsection{9轴融合}
9轴融合


% 参考文献
\newpage
% 将参考文献标题改为中文
% article 使用
\renewcommand\refname{参考文献}
% book 使用
%\renewcommand\bibname{参考文献}
\centering %居中
\bibliographystyle{plain}
\bibliography{main}

\end{document}

