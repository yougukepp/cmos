\documentclass[12pt,a4paper]{article}

\usepackage{xeCJK}
\usepackage{makeidx}
\usepackage[colorlinks,linkcolor=red]{hyperref}
\usepackage{xcolor}
\usepackage{listings}
\lstset{numbers=left,
    numberstyle=\tiny,
    keywordstyle=\color{blue!70}, commentstyle=\color{red!50!green!50!blue!50},
    frame=shadowbox,
    rulesepcolor=\color{red!20!green!20!blue!20}
}
\usepackage{multirow}

% 数学公式
\usepackage{amsmath}

%中文断行
\XeTeXlinebreaklocale "zh"
%段首缩进
\parindent 2em
\usepackage{indentfirst}
%文泉驿 字体
\setCJKmainfont{文泉驿等宽微米黑:style=Regular}
%\setCJKmainfont{文泉驿等宽正黑:style=Regular}
%楷体
%\setCJKmainfont{AR PL UKai CN:style=Regular}

% 将日期变为中文格式
\renewcommand{\today}{\number\year 年 \number\month 月 \number\day 日}

% 将目录标题改为中文
\renewcommand{\contentsname}{目录}

%制作索引
\makeindex
\printindex
\setcounter{secnumdepth}{5}

\title{姿态融合算法}
\author{彭鹏(QQ:516190948)}

\begin{document}
\maketitle
\newpage
\tableofcontents
\newpage

\section{术语表} 
以下是术语表:
\begin{table}[!hbp]
\begin{center}
    \begin{tabular}{|l|l|}
        \hline
        术语名 & 含义 \\
        \hline
        姿态 & 飞行器想对于N系的旋转 \\
        \hline
        四元数 & 表示姿态或者旋转 \\
        \hline
        估计姿态 & 陀螺仪积分计算后的值 \\
        \hline
        s系 & 传感器坐标系也可以认为是载体坐标系,xyz分别表示前右上 \\
        \hline
        n系 & 导航坐标系,xyz分别表示北东天 \\
        \hline
    \end{tabular}
    \caption{术语表\label{术语表}}
\end{center}
\end{table}

文中的陀螺仪、加速度计、磁力计都是三轴的。
\newpage

\section{概述}
四轴飞行器主要包括:姿态算法和控制算法两种算法。姿态算法完成飞行器自身姿态的计算,控制算法使用姿态计算出需要施加的控制量完成姿态控制,本文主要分析姿态算法。

姿态算法是四轴飞行器的核心算法,姿态算法的核心器件是陀螺仪,陀螺仪可以测量自身旋转的角速度。理论上只要陀螺仪精度足够,结合初始姿态通过积分可以计算飞行器任意时刻的姿态。然并卵,现实是残酷的,由于成本和尺寸的限制,四轴只能使用MEMS陀螺仪。MEMS陀螺仪精度不佳并有较大的漂移,在长时间的积分过程中会导致较大的积分误差,导致姿态随着时间累计而变得不可接受\cite{捷联惯导航}。

怎么办呢?\cite{9轴融合论文}联想轮船,陀螺仪类似舵手,舵手长时间工作后会将船带偏,这个时候需要了望手来不断纠偏。通常四轴中使用加速度计来充当了望手,纠偏轮船的方向。加速度计测量重力场的方向纠正陀螺仪积分得到的姿态,普通的四轴飞行器使用陀螺仪和加速度计就可以完成精度可接受的姿态解算。但是有一个问题,重力场方向竖直向下,可以纠正飞行器的水平倾斜,但飞行器绕重力场方向的旋转没法纠正。所以仅用6轴(陀螺仪3轴,加速度计3轴)融合的姿态只能保证飞行器的水平平衡,无法保证其航向稳定\footnote{会水平旋转},说通俗一点也就是飞行器会找不到北。类似加速度计的思路,传感器组合中加入一个测量地磁的传感器\cite{经典博客},就可以纠正飞行器航向上的偏移,这就是通常所说的9轴姿态融合。

\section{算法}
为了便于分析,飞行器停止时,使s系前右上与n系北东天分别重合。姿态的解算就是求取s系相对于n系旋转的角度,表示姿态的方法很多,本文使用欧拉角和四元数表示姿态。姿态解算的中间过程用四元数表示,姿态结算的结构用欧拉角表示。原因是欧拉角物理含义明确便于执行飞行控制,四元数做姿态计算可以降低运算量。
姿态解算有5个已知量作为算法的输入,他们分别是:
\begin{enumerate}
    \item 重力场

        与n系固连,恒定竖直向下,仅有天轴分量,是一个常矢量,与坐标系无关。
    \item 地磁场

        与n系固连,有一个北轴的分量和一个天轴的分量,但是没有东轴分量,是一个常矢量,与坐标系无关。
    \item 加速度计\footnote{加速度的工作原理可以参考加速度计原理\cite{加速度计原理}}数据

        该数据与s系固连,与重力场的方向相反\footnote{准确的说,加速度计的测量值叫比力加速度,表示作用在物体上非引力矢量和产生的加速度。如果飞行器受力不平衡会导致加速度计无法准确表征重力的反作用力的加速度,也就是说加速度计对干扰非常敏感,精确的姿态融合算法需要給加速度计滤波。},可部分\footnote{水平旋转角无法表征}表征姿态。
    \item 磁力计数据

        该数据与s系固连,指向地磁场的方向,可部分\footnote{绕地磁场轴的旋转无法表征}表征姿态。
    \item 陀螺仪数据

        与s系固连,表征s系旋转的角速度,可独立积分后估计姿态。
\end{enumerate}
姿态解算的输出有三个量,分别为\footnote{欧拉角与旋转顺序有关,本文使用航空次序角,也就是n系的天东北}:
\begin{enumerate}
    \item 偏航角$\psi$
        绕天轴旋转的角度
    \item 俯仰角$\theta$
        绕东轴旋转的角度
    \item 横滚角$\phi$
        绕北轴旋转的角度
\end{enumerate}
利用$\psi,\theta,\phi$可以计算出需要施加的控制量,完成姿态控制。

姿态融合算法主要分为三个函数实现
\begin{enumerate}
    \item 3轴姿态

        使用陀螺仪的数据和四元数微分方程 用积分法获取估计姿态
    \item 6轴融合

        使用陀螺仪、加速度计的数据和梯度下降算法
    \item 9轴融合

        使用陀螺仪、加速度计的数据和梯度下降算法
\end{enumerate}
下面对三个算法分别阐述。

\subsection{3轴姿态}
3轴姿态使用陀螺仪的数据\footnote{角速度}计算姿态的微分,然后利用微分求积分获取当前姿态。用四元数表示表示向量,将其标量部分插入零即可。所以陀螺仪提供的数据为:
\begin{eqnarray}
    ^s\omega=[0\quad\omega_{x}\quad\omega_{y}\quad\omega_{z}] \label{陀螺仪数据}
\end{eqnarray}

公式中$^s\omega$表示在传感器坐标系中陀螺仪测到的角速度.
$\omega_{x},\omega_{y},\omega_{z}$分别表示飞行器绕s系的前、右、上轴的角速度。

四元数微分方程为:
\begin{eqnarray}
    ^s_n\dot{q}_{\omega,t}=\frac{1}{2}{^s_n\hat{q}_{est,t-1}}\otimes^s\omega_t \label{四元数微分方程}
\end{eqnarray}

四元数微分方程使用就行,这里不推导了,笔者也不会。
$^s_n\dot{q}_{\omega,t}$表示t时刻从s系旋转到n系的四元数\footnote{可以理解为四元数是一个时间的函数$^s_n{q}=func(t)$,$^s_n\dot{q}=func'(t)$}微分\footnote{下标$\omega$表示该值通过陀螺仪输出的角速度算出},${^s_n\hat{q}_{est,t-1}}$表示t-1\footnote{上次迭代的时刻}时刻姿态的估计值,也就是s系到n系四元数的估计值\footnote{est是estimation缩写}。$\otimes$表示四元数乘法

四元数写为分量形式:
\begin{eqnarray}
    \hat{q}=[\hat{q}_0\quad \hat{q}_1\quad \hat{q}_2\quad \hat{q}_3] \label{分量四元数}
\end{eqnarray}

在进行四元数微分方程分量化前定义虚数运算公式:
\begin{eqnarray}
    i^2 = j^2 = k^2 = -1
\end{eqnarray}
\begin{eqnarray}
    ij = -ji = k
\end{eqnarray}
\begin{eqnarray}
    jk = -kj = i
\end{eqnarray}
\begin{eqnarray}
    ki = -ik = j
\end{eqnarray}

四元数微分方程分量化:
\begin{equation} \label{微分方程分量化}
    \begin{aligned}
        {^s_n\hat{q}_{est,t-1}}\otimes^s\omega_t
        &= [\hat{q}_0\quad i\hat{q}_1\quad j\hat{q}_2\quad k\hat{q}_3] \otimes [0\quad i\omega_{x}\quad j\omega_{y}\quad k\omega_{z}] \\
        &= \hat{q}_0 \cdot 0 + i    \hat{q}_0 \omega_x + j    \hat{q}_0 \omega_y + k    \hat{q}_0 \omega_z \\
      &+ i \hat{q}_1 \cdot 0 + (ii) \hat{q}_1 \omega_x + (ij) \hat{q}_1 \omega_y + (ik) \hat{q}_1 \omega_z \\
      &+ j \hat{q}_2 \cdot 0 + (ji) \hat{q}_2 \omega_x + (jj) \hat{q}_2 \omega_y + (jk) \hat{q}_2 \omega_z \\
      &+ k \hat{q}_3 \cdot 0 + (ki) \hat{q}_3 \omega_x + (kj) \hat{q}_3 \omega_y + (kk) \hat{q}_3 \omega_z \\
        &=   i \hat{q}_0 \omega_x + j \hat{q}_0 \omega_y + k \hat{q}_0 \omega_z \\
        &  -   \hat{q}_1 \omega_x + k \hat{q}_1 \omega_y - j \hat{q}_1 \omega_z \\
        &  - k \hat{q}_2 \omega_x -   \hat{q}_2 \omega_y + i \hat{q}_2 \omega_z \\
        &  + j \hat{q}_3 \omega_x - i \hat{q}_3 \omega_y -   \hat{q}_3 \omega_z \\
        &= -(\hat{q}_1 \omega_x + \hat{q}_2 \omega_y + \hat{q}_3 \omega_z) \\
        &+ i(\hat{q}_0 \omega_x + \hat{q}_2 \omega_z - \hat{q}_3 \omega_y) \\
        &+ j(\hat{q}_0 \omega_y - \hat{q}_1 \omega_z + \hat{q}_3 \omega_x) \\
        &+ k(\hat{q}_0 \omega_z + \hat{q}_1 \omega_y - \hat{q}_2 \omega_x) \\ 
        =\left[\begin{array}{c}
                \dot{q}_{0} \\
                \dot{q}_{1} \\
                \dot{q}_{2} \\
                \dot{q}_{3}
        \end{array}\right]
        &=\left[\begin{array}{c}
        - \hat{q}_1 \omega_x - \hat{q}_2 \omega_y - \hat{q}_3 \omega_z \\
          \hat{q}_0 \omega_x + \hat{q}_2 \omega_z - \hat{q}_3 \omega_y \\
          \hat{q}_0 \omega_y - \hat{q}_1 \omega_z + \hat{q}_3 \omega_x \\
          \hat{q}_0 \omega_z + \hat{q}_1 \omega_y - \hat{q}_2 \omega_x
        \end{array}\right]
    \end{aligned}
\end{equation}

整理式\ref{四元数微分方程}和式\ref{微分方程分量化},可得微分方程分量形式:
\begin{eqnarray}
    \begin{aligned}
    ^s_n\dot{q}_{\omega,t}&=\frac{1}{2}{^s_n\hat{q}_{est,t-1}}\otimes^s\omega_t
    =\frac{1}{2}
    \left[\begin{array}{c}
        - \hat{q}_1 \omega_x - \hat{q}_2 \omega_y - \hat{q}_3 \omega_z \\
          \hat{q}_0 \omega_x + \hat{q}_2 \omega_z - \hat{q}_3 \omega_y \\
          \hat{q}_0 \omega_y - \hat{q}_1 \omega_z + \hat{q}_3 \omega_x \\
          \hat{q}_0 \omega_z + \hat{q}_1 \omega_y - \hat{q}_2 \omega_x
    \end{array}\right] \\
    &=\frac{1}{2}
    \left[ \begin{array}{rrr}
            -\hat{q}_{1,t-1} & -\hat{q}_{2,t-1} & -\hat{q}_{3,t-1} \\
             \hat{q}_{0,t-1} & -\hat{q}_{3,t-1} &  \hat{q}_{2,t-1} \\
             \hat{q}_{3,t-1} &  \hat{q}_{0,t-1} & -\hat{q}_{1,t-1} \\
            -\hat{q}_{2,t-1} &  \hat{q}_{1,t-1} &  \hat{q}_{0,t-1}
    \end{array} \right]
    \left[\begin{array}{c}
            \omega_{x} \\
            \omega_{y} \\
            \omega_{z}
    \end{array}\right] \label{微分最终式}
    \end{aligned}
\end{eqnarray}

式\ref{微分最终式}中四元数上方的三角表示估计值,上方点表示微分,下标t表示当前迭代时刻,下标t-1表示上次迭代时刻,下标0、1、2、3表示四元数分量编号。

四元数积分方程为:
\begin{eqnarray}\label{四元数积分方程}
    ^s_n\hat{q}_{\omega,t}&={^s_n\hat{q}_{est,t-1}}+^s_n\dot{q}_{\omega,t}\Delta t
\end{eqnarray}

式\ref{四元数积分方程}中$\Delta t$表示迭代周期。

整理上文的推导,可以的出三轴姿态算法的步骤:
\begin{enumerate}
    \item 姿态初始化

        算法启动时使s系前右上与n系北东天分别重合。所以欧拉角:
        \begin{equation}\label{初始欧拉角}
            \psi = \theta = \phi = 0
        \end{equation}
        %
        % q0 = cos(c/2)cos(b/2)cos(a/2) + sin(c/2)sin(b/2)sin(a/2)
        % q1 = sin(c/2)cos(b/2)cos(a/2) - cos(c/2)sin(b/2)sin(a/2)
        % q2 = cos(c/2)sin(b/2)cos(a/2) + sin(c/2)cos(b/2)sin(a/2) 
        % q3 = cos(c/2)cos(b/2)sin(a/2) + sin(c/2)sin(b/2)cos(a/2)
        % 初始化时 a b c 都为0带入上公式 所以初始化为 s_quaternion[4] = {1,0,0,0}
        %利用式\ref{初始欧拉角}和式\ref{欧拉角转四元数},得出四元数初始值为\footnote{研究推导}:
        利用式\ref{初始欧拉角}和式,得出四元数初始值为\footnote{推导欧拉角转四元数公式}:
        \begin{equation}
            q=[1\quad 0\quad 0\quad 0] \label{分量四元数}
        \end{equation}
    \item 微分

        整理式\ref{微分最终式},得出微分迭代方程:

        \begin{equation} \label{微分迭代方程}
            \begin{aligned} 
                &\dot{q}_{0,t} = -\hat{q}_{1,t-1} \omega_x - \hat{q}_{2,t-1} \omega_y - \hat{q}_{3,t-1} \omega_z \\
                &\dot{q}_{1,t} =  \hat{q}_{0,t-1} \omega_x - \hat{q}_{3,t-1} \omega_y + \hat{q}_{2,t-1} \omega_z \\
                &\dot{q}_{2,t} =  \hat{q}_{3,t-1} \omega_x + \hat{q}_{0,t-1} \omega_y - \hat{q}_{1,t-1} \omega_z \\
                &\dot{q}_{3,t} = -\hat{q}_{2,t-1} \omega_x + \hat{q}_{1,t-1} \omega_y + \hat{q}_{0,t-1} \omega_z
            \end{aligned}
        \end{equation} 
        式\ref{微分迭代方程}中四元数上方的点表示四元数微分\footnote{当前迭代想对于上次迭代时的变化量},下标t表示当前迭代时刻,下标t-1表示上次迭代时刻,下标0、1、2、3表示四元数分量编号,$x,y,z$代表s系前右上
    \item 积分

        整理式\ref{四元数积分方程}和\ref{微分迭代方程},积分即可求得当前姿态。
        \begin{equation} \label{积分迭代方程}
            \begin{aligned} 
                &\dot{q}_{0,t} = \hat{q}_{0,t-1} + \dot{q}_{0,t} \Delta t \\
                &\dot{q}_{1,t} = \hat{q}_{1,t-1} + \dot{q}_{1,t} \Delta t \\
                &\dot{q}_{2,t} = \hat{q}_{2,t-1} + \dot{q}_{2,t} \Delta t \\
                &\dot{q}_{3,t} = \hat{q}_{3,t-1} + \dot{q}_{3,t} \Delta t
            \end{aligned}
        \end{equation} 
        式\ref{积分迭代方程}中上下标含义与式\ref{微分迭代方程}相同
\end{enumerate}

由以上三步即可利用陀螺仪完成3轴姿态算法。

\section{后续工作}
\begin{enumerate}
    \item 考虑“指东针“

        将重力场矢量与地磁叉乘得到一个仅有东向\footnote{或向西,由叉乘顺序决定}分量的矢量,这样可以简化计算。但有两个问题需要验证:
        \begin{enumerate}
            \item 叉乘的代价与简化的计算量相比是否值得;
            \item 叉乘后该量比不是磁力计独立测试的数据,与加速度计有一定的相关性,是否会引入误差。
        \end{enumerate} 
    \item 卡尔曼滤波

        温度计和房间温度的例子
    \item 加速度标定

        取平均或最小二乘法
    \item 控制算法

        PID算法
\end{enumerate}

% 参考文献
\newpage
% 将参考文献标题改为中文
% article 使用
\renewcommand\refname{参考文献}
% book 使用
%\renewcommand\bibname{参考文献}
\centering %居中
\bibliographystyle{plain}
\bibliography{姿态融合算法}

\end{document}

