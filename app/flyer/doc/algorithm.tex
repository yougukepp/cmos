% section — subsection — subsubsection — paragraph — subparagraph

\section{算法步骤}\label{section:最终结论}
下面总结3轴融合,6轴融合,9轴融合的步骤。
\subsection{3轴融合}
3轴融合算法可以分解为一下步骤:
\begin{enumerate}
    \item 四元数初始化

        积分操作需要初始值,而四元数的几何意义不是很明确,难以确定其初始值.故四元数的初始值需要通过欧拉角计算.初始欧拉角全为零,代入式\ref{欧拉角转四元数},可以得到四元数的初值.
    \item 微积分求四元数

        通过陀螺仪输出的角速度求四元数微分,利用该微分值求四元数.四元数微积分公式为式\ref{四元数递推分量方程}.
    \item 计算瞬时欧拉角

        为了和加计,磁计融合以及便于后面的姿态控制,四元数需要转换为有物理含义的欧拉角.四元数转欧拉角使用公式\ref{四元数转欧拉角}.
\end{enumerate}

\subsection{6轴融合}
6轴融合在3轴融合的基础上,仅使用加计数据校验陀螺仪,加计通过测量比力测试重力.本算法假定欧拉角旋转顺序为$z$-$y$-$x$,所以姿态解算时旋转顺序相反,为$x$-$y$-$z$.则首先绕$x$轴反向旋转$\theta$,如图\ref{横滚角补偿}:
\begin{figure}[h]
\begin{center}
    \includegraphics[height=5cm, width=5cm]{fig/6axis_x.pdf}
    \caption{横滚角补偿}\label{横滚角补偿}
\end{center}
\end{figure}

由图\ref{横滚角补偿}得横滚角为:
\begin{equation}\label{加计横滚角}
    \theta = \arctan2(y_2,z_2) = \arctan2(a_y,a_z)
\end{equation} 

其中$a_y$   表示加计$y$分量,
$a_z$       表示加计$z$分量.
\begin{figure}[h]
    \begin{center}
        \includegraphics[height=5cm, width=5cm]{fig/6axis_y.pdf}
        \caption{俯仰角补偿}\label{俯仰角补偿}
    \end{center}
\end{figure}

由图\ref{俯仰角补偿}得俯仰角角为:
\begin{equation}\label{加计俯仰角}
    \phi = \arctan2(x_2,z_2) = \arctan2(a_x,a_z)
\end{equation} 

其中$a_x$   表示加计$x$分量,
$a_z$       表示加计$z$分量.

下面总结六轴融合步骤:
\begin{enumerate}
    \item 获取估计欧拉角

        利用式\ref{四元数转欧拉角},求取当前的估计欧拉角.
    \item 获取当直接欧拉角

        利用式\ref{加计横滚角},\ref{加计俯仰角}分别求取当前横滚角$\theta$,俯仰角$\phi$
    \item 水平融合

        使用直接欧拉角修正估计欧拉角,计算公式如下:
        \begin{equation}\label{水平融合}
            \left\{\!\!\!\begin{array}{ll}
                    \theta = \hat{\theta} + \alpha(\theta_a - \hat{\theta}), & \alpha \in [0, 1] \\
                    \phi   = \hat{\phi} + \beta(\phi_a - \hat{\phi}), & \beta \in [0, 1]
                \end{array}\right.
            \end{equation} 
            
            其中$\hat{x}$   表示$x$的估计值, 
            $x_a$           表示加计获得的$x$指,
            $\alpha$,$\beta$分别表示横滚角,俯仰角的融合系数.
    \item 更新姿态

        由于算法内部的姿态表示为四元数,所以水平融合完成后需要利用式\ref{欧拉角转四元数}将欧拉角\footnote{由于加计无法修正偏航角,所以需要令欧拉角中的偏航角为估计值.}转换回四元数,并更新四元数.
\end{enumerate}

\subsection{9轴融合}
9轴融合在6轴融合的基础上,使用磁计数据修正偏航角.而修正偏航角的不变矢量本算法选用"指东针".下面先介绍该矢量,再总结算法步骤.
\subsubsection{指东针}
指东针的定义如图\ref{指东针},
\begin{figure}[h]
\begin{center}
    \includegraphics[height=5cm, width=5cm]{fig/eastor.pdf}
    \caption{指东针}\label{指东针}
\end{center}
\end{figure}
其中重力叉乘磁力获取指向正东的不变矢量.

\begin{equation}\label{指东针叉乘}
    \bfx{e} = \bfx{a} \otimes \bfx{m}
\end{equation} 

式\ref{指东针叉乘}中$\bfx{a}$   表示加计测量值,
$\bfx{m}$       表示磁计测量值,
$\bfx{e}$       表示指东针.
9轴融合在6轴融合的基础上,使用磁计数据\footnote{结合加计数据}修正偏航角.指东针修正偏航角如图\ref{偏航角补偿}.
\begin{figure}[h]
\begin{center}
    \includegraphics[height=5cm, width=5cm]{fig/9axis.pdf}
    \caption{偏航角补偿}\label{偏航角补偿}
\end{center}
\end{figure}
由图\ref{偏航角补偿}得俯仰角角为:
\begin{equation}\label{磁计偏航角}
    \psi = \arctan2(x_2,y_2) = \arctan2(a_x,a_y)
\end{equation} 

其中$a_x$   表示指东针$x$分量,
$a_y$       表示指东针$y$分量.
下面总结9轴融合算法步骤:
\begin{enumerate}
    \item 获取估计偏航角

        利用式\ref{四元数转欧拉角},求取当前的估计偏航角.
    \item 获取当直接偏航

        利用式\ref{磁计偏航角}求取当前偏航角$\psi$.
    \item 航向融合

        使用直接偏航角修正估计偏航角,计算公式如下:
        \begin{equation}\label{偏航融合}
            \begin{array}{ll}
                    \psi   = \hat{\psi} + \gamma(\psi_a - \hat{\psi}), & \gamma \in [0, 1]
            \end{array}
        \end{equation} 
            
            其中$\hat{x}$   表示$x$的估计值, 
            $x_a$           表示加计获得的$x$指,
            $\gamma$        表示偏航角的融合系数.
    \item 更新姿态

        类似6轴融合,由于算法内部的姿态表示为四元数,所以偏航融合完成后需要利用式\ref{欧拉角转四元数}将偏航角\footnote{由于磁计无法修正水平姿态,所以需要令欧拉角中的横滚角,俯仰角分别为估计值.}转换回四元数,并更新四元数.
\end{enumerate}

至此算法步骤分析完毕,下面总结后面需要的工作.

